\clearpage
\chapter{Concluding remarks}
\label{ch:conclusion}



Normalisation and housekeeping genes.
Application to redo all the analysis.
Metabolic genes that are correlated \mRNAs{}/proteins (not same impact when different data and
paired-data).




Creating atlas of normal// or unified transcriptomics was also a good way to
\textbf{play around} for later on justify for example all the cell lines atlas
(normal or not). En gros : j'ai macheté la foret et maintenant on a une meilleure
idée de comment approcher  l'ensemble.

A termes, on pourrait envisager que quiquonque crée une nouvelle lignée cellulaire,
doivent la caractériser (SNP,CNV and RNAseq).


And more than that, personally this was the greatest opportunity to pursue.



EBI or alike \rightarrow\ need money to provide a full Atlas uniformed pipeline.

At term we will have a transcriptome of reference like we have a reference genome.





\section{Discussion}

\textbf{Normalisation +++}
Tout ce qui a de bonnes correlations
(entre \mRNAs\ ou proteins ou entre \mRNAs/proteins)
est probablement moins régulé.
So, when someone want to plan a new experiment
and they are going to use or work on only a few proteins,
it is important to include proteins from different types
(for example, not only for proteins from metabolic pathways).


\cite{De_Lichtenberg2005-ka}
(extracted from~\cite{Liu2016-re} review)
have found in yeast that
the regulation of a limited number of key components of
complex proteins is enough to regulate protein complex
instead of of synthesizing and degrading everything.
So, need to check for the \mRNAs\ ratios of the complexes
if they are clear differences. Do the same for the proteins.
Finally check and comopare the ratios of proteins/\mRNAs\ within each of the complexes.

















\section{Have to add before the submission}
\begin{itemize}
 \item Tables of the correlation within the 8 datasets
 \item Tables for the correlation for \setOne, and \setTwo. (all correlation
     values for all the corresponding tissues pairs)
 \item For all the groups and classes in \Cref{ch:Transcriptomics} give the list
     of the consensus genes.
\end{itemize}
