\chapter{Supplementary for Chapter 3}
\label{ch:SupplExpress}

\begin{figure}[htbp]
    \centering
    \begin{subfigure}[b]{0.45\textwidth} % "0.45" donne ici la largeur de l'image
        \centering \includegraphics[width=\textwidth]{expressed/histogramCastle_noLog.pdf}
        \caption{}\label{fig:histCastle_nolog}
    \end{subfigure}
~ % ce symbole ajoute un espacement horisontal entre les premières deux images
    \begin{subfigure}[b]{0.45\textwidth} % "0.45" donne ici la largeur de l'image
          \centering \includegraphics[width=\textwidth]{expressed/histogramBrawand_noLog.pdf}
          \caption{}\label{fig:histBrawand_nolog}
      \end{subfigure}
\caption{Untransformed FPKM distribution across the
    datasets}\label{fig:unscaledDistrib}
\end{figure}

\begin{figure}
    \includegraphics[scale=0.40]{expressed/AnscombePanel.pdf}\centering
    \caption[Anscombe quartet --- why data should always visualy checked]
    {\label{fig:Anscombe}\textbf{Anscombe quartet --- why data should always
    visually checked.}\smallbreak{} All the datasets, while presenting
    different distribution, have equal or very similar descriptive statistic
    indicators. Their means and variances (for both $x$ and $y$ variables),
    their Pearson correlation between $x$ and $y$,
    as their linear regressions are very similar when not equal.}
\end{figure}


\section{Correlation}\label{sec:CorrMore}

While $1$ and $-1$ mean a perfect correlation (either positive or negative),
a value equals to $0$ expresses that the two variables are independent.
A value within $\mathopen]-1,0\mathclose[$
or $\mathopen]0,1\mathclose[$ needs more interpretation. In biology, if the
coefficient is within $[-0.5,0.5]$, the variables are often considered as
independent.

\subsection{Spearman correlation}\label{subsec:SpearmanCor}
The Spearman correlation coefficient (usually noted as $\rho$)
is more robust than the Pearson correlation.
However, it only assesses the monotonic dependence between two variables.
The Spearman correlation is equal to the Pearson correlation of the value
\emph{ranks} between two variables.

\subsection{Pearson correlation}\label{subsec:PearsonCor}
The Pearson correlation coefficient (usually noted as $r$) assesses the linear
dependence between two variables. It is invariant to systematic addition of a
constant or to simple scaling factors between the two variables.

