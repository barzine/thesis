\usepackage{scrhack}
\usepackage[dottedtoc,eulerchapternumbers]{classicthesis}
%\usepackage{arsclassica}
\usepackage{tocbasic}
\usepackage{polyglossia}
\setdefaultlanguage[variant=uk]{english}
%\setmainlanguage[variant=uk]{english}
\usepackage[english, iso]{isodate}
\usepackage{hyperref}
\usepackage[natbib=true,backend=biber,style=authoryear,maxcitenames=1,maxbibnames=99,useprefix,backref]{biblatex} %add backref for draft
\usepackage{minted}
%\usepackage[explicit]{titlesec}
\usepackage{csquotes}
\usepackage{epigraph}
%\renewcommand{\epigraphflush}{center}
\usepackage{booktabs,dcolumn}
\usepackage{microtype}
\usepackage{rotating}
\usepackage{multirow}
%\usepackage[table,xcdraw]{xcolor}
\usepackage{relsize}
\usepackage[onehalfspacing]{setspace}
%\usepackage{setspace}
%\usepackage[multidot]{grffile}
\usepackage{marginnote}
\usepackage{float}
\usepackage{fontspec}
\usepackage{comment}
\usepackage{amsmath}
\usepackage{unicode-math}
\usepackage{xpatch}
\usepackage{xltxtra}
\usepackage{xcolor}
\usepackage{calc}
\usepackage{ragged2e}
\usepackage[font=bf,format=hang,labelfont={bf,normal},textfont=normal,up,margin=5pt,labelformat=default,labelsep=period]{caption}
%format=hang
%\usepackage[font=small,format=plain,labelfont=bf,up,textfont=normal,up,justification=justified,singlelinecheck=false,margin=5pt]{caption}
\usepackage{subcaption}
%\usepackage[user,xr]{zref} %for cross reference between files

%\usepackage{ifoddpage}
\usepackage{afterpage}
\usepackage{verbatim}
\usepackage{xspace} %the original author doesn't recommend it
\usepackage{pgfplots}
\pgfplotsset{compat=1.14}
\usepackage{expdlist}
\usepackage{pdflscape}
\usepackage{longtable}
\usepackage{eqlist}
\usepackage{tracklang}
\usepackage{mfirstuc}
\usepackage{etoolbox}
\usepackage{xkeyval}
\usepackage{textcase}
\usepackage{xfor}
\usepackage{datatool-base}
\usepackage{amsgen}
\usepackage{interval}
\usepackage{ragged2e}

\usepackage%[style=long,nolist] %as advised in classicthesis man
[xindy, toc, nopostdot,style=long,nolist]
{glossaries} %%hyperref has to be already loaded
\usepackage[automake,nomain,acronym]{glossaries-extra}
\setglossarystyle{longheader}
%\setabbreviationstyle[acronym]{long-short}
\makeglossaries%
%\loadglsentries{glossary.tex}

\usepackage{bookmark}
\usepackage{geometry}%% for page layout
\usepackage[nameinlink, noabbrev]{cleveref}


\setmainfont[Mapping=tex-text, Numbers=OldStyle]{TeX Gyre Pagella}
%\setmainfont[Ligatures={NoRequired,NoCommon,NoContextual}]{Font Name}
%\setsansfont[Mapping=tex-text, Numbers=OldStyle]{Droid Sans}
%\setmonofont{Droid Sans Mono}

%\setmainfont[Ligatures=TeX, Numbers=OldStyle]{GaramondNo8}

%\setmainfont[Ligatures=TeX, Numbers=OldStyle]{TeX Gyre Pagella}
\setmainfont[Ligatures=NoCommon]{Linux Libertine O}

%\setmainfont[Ligatures=TeX, Numbers=OldStyle]{Helvetica}

\setsansfont[
    Ligatures=TeX,
    UprightFont={* Light},
    BoldFont={*},
    ItalicFont={* Light Italic},
    BoldItalicFont={* Italic},
    Scale=MatchLowercase
]{Gill Sans}
\setmonofont[Scale=MatchLowercase]{WenQuanYi Zen Hei Mono}
\setmathfont[
    Extension=.otf,
    BoldFont=*bold,
]{xits-math}


%%%% Check with classicthesis manual to redifine the chapter and sections formats
%\titleformat{\chapter}[display]%
%{\relax}{\mbox{}\oldmarginpar{\vspace*{-3\baselineskip}\color{halfgray}\chapterNumber\thechapter}}{0pt}%
%{\raggedright\spacedallcaps}[\Huge\vspace*{.8\baselineskip}\titlerule]
%%sections
%\titleformat{\section}
%{\relax}{\textbf{\MakeTextLowercase{\thesection}}}{1em}{\spacedlowsmallcaps}
%%subsections
%\titleformat{\subsection}{\relax}{\textsc{\MakeTextLowercase{\thesubsection}}}{1em}{\normalsize}
%%subsubsections
%\titleformat{\subsubsection}{\relax}{\textsc{\MakeTextLowercase{\thesubsubsection}}}{0.5em}{\normalsize}
%% need to take care of the numbering of the captions if used.

\setkomafont{title}{\rmfamily\Huge}
\setkomafont{subject}{\normalfont\normalcolor}
\isodash{\ensuremath{\cdot}}% Replacement for `‧` = U+‧2027 HYPHENATION POINT

\definecolor{cambridgeblue}{RGB}{163, 193, 173}
\definecolor{cambridgebluedark}{RGB}{17, 94, 103}


%\defaultfontfeatures{Ligatures=TeX,Numbers=OldStyle}

\addbibresource{Bibliography.bib}
\addbibresource[label=ownpubs]{Publications.bib}


\setstretch{1.1}
\graphicspath{{gfx/}} %as instructed in classicthesis

%\graphicspath{{gfxRasterOnly/}} %as instructed in classicthesis
% For debugging: showframe=true to see the layout frames
\geometry{bottom=65mm,showframe=false,margin=35mm,marginparsep=3mm,marginparwidth=20mm}




\pagestyle{scrheadings}


%%%%%%%
%%% customisation stollen from Manuel Kuehner template
%%some calculations
%% calc package is needed which is loaded here: 01_Preamble/CommonPackages.tex
%% If you want to understand the calculations visit:
%% http://en.wikibooks.org/wiki/LaTeX/Page_Layout
\newlength{\myLenghthFootAbstand}
\setlength{\myLenghthFootAbstand}{\paperheight-1in-\topmargin- \headheight-\headsep-\textheight-\footskip}
\newlength{\myLenghthTemp}
\setlength{\myLenghthTemp}{\myLenghthFootAbstand+\baselineskip}

\clearscrheadfoot%
% Header
\ohead{%
    \headmark%
    }
% Left (even page numbers) footer
\lefoot%
[% scrplain style (begin)
    \setlength{\unitlength}{\myLenghthFootAbstand}%
    \begin{picture}(0,0)%
        \put(0,-1)%
        {%
            \makebox(0,0)[lb]%
            {%
                \rule{0.4pt}{\myLenghthTemp}%
            }%
        }%
    \end{picture}
    \llap{\pagemark~~~~~}
]% scrplain style (end)
%
{% scrheadings style (begin)
    \setlength{\unitlength}{\myLenghthFootAbstand}%
    \begin{picture}(0,0)%
        \put(0,-1)%
        {%
            \makebox(0,0)[lb]%
            {%
                \rule{0.4pt}{\myLenghthTemp}%
            }%
        }%
    \end{picture}
    \llap{\pagemark~~~~~}%
    }% scrheadings style (end)

%% Right (odd page numbers) footer
\rofoot%
[% scrplain style (begin)
    \rlap{~~~~~\pagemark}%%
    \setlength{\unitlength}{\myLenghthFootAbstand}%
    \begin{picture}(0,0)%
        \put(0,-1)%
        {%
            \makebox(0,0)[lb]%
            {%
                \rule{0.4pt}{\myLenghthTemp}%
            }%
        }%
    \end{picture}%
    ]% scrplain style (end)
    %
{% scrplain style (begin)
    \rlap{~~~~~\pagemark}%%
    \setlength{\unitlength}{\myLenghthFootAbstand}%
    \begin{picture}(0,0)%
        \put(0,-1)%
        {%
            \makebox(0,0)[lb]%
            {%
                \rule{0.4pt}{\myLenghthTemp}%
            }%
        }%
    \end{picture}%
    }% scrplain style (end)





\setlength{\headheight}{1.1\baselineskip}
%%%%%%%


% %%% More layout definitions
%    \definecolor{primary}{HTML}{429EB6}
%    \definecolor{secondary}{HTML}{DE8950}
%    \definecolor{tertiary}{HTML}{4DB966}
%    \definecolor{quarternary}{HTML}{F3B5FB}
%    \definecolor{quinary}{HTML}{E7BE05}
%    \newcommand*\primaryname{blue}
%    \newcommand*\secondaryname{orange}
%    \newcommand*\tertiaryname{green}
%    \newcommand*\quarternaryname{purple}
%    \newcommand*\quinaryname{yellow}
%
%    \colorlet{thesis@toccolor}{black}
%    \colorlet{thesis@linkcolor}{black!70}
%
%    \definecolor{codenormal}{HTML}{404040}
%    \AtBeginEnvironment{minted}{\color{codenormal}}
%    \usemintedstyle{klmrthesis}

%   \hypersetup{%
%        hypertexnames=false,
%        linktoc=all,
%        colorlinks,
%        citecolor=thesis@linkcolor,
%        linkcolor=thesis@linkcolor,
%        filecolor=thesis@linkcolor,
%        urlcolor=thesis@linkcolor
%    }


%%%%%%%
%\newcounter{dataset}[chapter]
%\newenvironment{dataset}[1][]{\refstepcounter{dataset}\par\medskip
%   \textbf{Dataset~\thedataset. #1} \rmfamily}{\medskip}

%%%%% new custom list (of datasets)
%\DeclareNewTOC[%
%    type=dataset,%
%    types=datasets,% used in the \listof.. command
%    chapter,% define a chapter environment
%    name=Dataset,%
%    listname={List of datasets}%
%    ]{lod}

%%%have a look at http://tex.stackexchange.com/questions/198932/list-of-newcounter/198957#198957
%%%% and at http://tex.stackexchange.com/questions/6478/new-figure-environment/96493#96493


%%%% custom macros
\newcommand*\captitle[1]{\textbf{#1}}
\setkomafont{caption}{\bfseries\sffamily}

%if use of classicthesis
\newcommand*\todo[1]{%
    \graffito{\textcolor{red}{TO\ DO:~#1}}}

%if no use of classicthesis but \usepackage{scrpage2} instead
%\newcommand*\todo[1]{%
%    \comment{TO\ DO:#1}}

\newcommand{\fixme}[1]{{\let\marginpar\oldmarginpar\todo{#1}}}

\newcommand*\TK[1]{
    \textcolor{blue}{[TK:\ #1]}}

\newcommand*\Rough[1]{
    Speak about:
    \textsl{\textcolor{gray}{#1}}}

\newcommand*\rough[1]{\textsl{\textcolor{gray}{#1}}}

%%%% shortcuts
\newcommand*\myPlace{Cambridge (UK)}
\newcommand*\myDate{21 November 2016}
\newcommand*\myName{Mitra Parissa Barzine}

%% specification
\newcommand*\latin[1]{\textit{#1}}
\newcommand*\mol[1]{\textnormal{#1}}
\newcommand*\gene[1]{\textit{#1}}
\newcommand*\ko[1]{\textit{#1\textsuperscript{\(-/-\)}}}
\newcommand*\protein[1]{#1}
\newcommand*\species[1]{\textit{#1}}
\newcommand*\ribo[1]{\textnormal{#1}}
\newcommand*\dataset[1]{\textnormal{#1}}
\newcommand*\paper[1]{\emph{#1}}
\newcommand*\tissue[1]{\emph{#1}}
\newcommand*\soft[1]{\emph{#1}}
\newcommand*\comp[1]{\texttt{#1}}
\newcommand*\frfig[1]{\textnormal{#1}} %for changing element of a figure in the main text


%% identifiers
\newcommand*\ENA[1]{\gls{ENA} \textnormal{#1}}
\newcommand*\ArrayExpress[1]{\gls{ArrayExpress} \textnormal{#1}}
\newcommand*\dbGaP[1]{\gls{dbGaP} \textnormal{#1}}
\newcommand*\Pride[1]{\gls{Pride} \textnormal{#1}}
\newcommand*\Proteomicsdb[1]{\gls{Proteomicsdb} \textnormal{#1}}


%% personal abbr.
\newcommand*\TKR{\TK{add reference}}
\newcommand*\etal{\textit{et al.}}
\newcommand*\eg{\textit{e.g.}}
\newcommand*\ie{\textit{i.e.}}
\newcommand*\rew[1]{\hl{#1}}
\newcommand*\sepfootnote{$^{,}$}
\newcommand*\mycheckmark{\Large ✔}
\newcommand*\mynormalcheckmark{✔}
\newcommand*\NB{\textbf{N.B.}: }

%\newcommand*\mRNAs{\mol{mRNAs}}
%\newcommand*\mRNA{\mol{mRNA}}
%\newcommand*\DNA{\mol{DNA}}
\newcommand*\RNA{\gls{RNA}}
\newcommand*\mRNA{\gls{mRNA}}
\newcommand*\mRNAs{\glspl{mRNA}}
\newcommand*\DNA{\gls{DNA}}
\newcommand*\RPKM{\gls{RPKM}}
\newcommand*\FPKM{\gls{FPKM}}
\newcommand*\FPKMs{\glspl{FPKM}}
\newcommand*\Dnaseq{\gls{DNA-Seq}}
\newcommand*\dNTPs{\glspl{dNTP}}
\newcommand*\dNTP{\gls{dNTP}}
\newcommand*\Rnaseq{\gls{RNA-Seq}}
\newcommand*\ms{\gls{MS}}
\newcommand*\orbi{Orbitrap™}

\newcommand*\trep{\gls{TREP}}
\newcommand*\treps{\glspl{TREP}}


\newcommand*\Gtex{\gls{GTEx}}
\newcommand*\EBI{\gls{EBI}}
\newcommand*\egxa{\EBI\ Gene Expression Atlas}

%%%% datasets
\newcommand*\ibm{\dataset{\gls{IBM}}}
\newcommand*\vt{\dataset{Brawand}}
\newcommand*\brawand{\vt}
\newcommand*\uhlen{\dataset{Uhlén}}
\newcommand*\castle{\dataset{Castle}}
\newcommand*\gtex{\dataset{\Gtex}}
\newcommand*\pandey{\dataset{Pandey}}
\newcommand*\cutler{\dataset{Cutler}}
\newcommand*\kuster{\dataset{Kuster}}
%%%% tools
\newcommand*\irap{\soft{iRAP}}
\newcommand*\htseq{\soft{HTSeq-count}}
\newcommand*\cuffl{\soft{Cufflinks2}}
\newcommand*\toph{\soft{TopHat2}}

\newcommand*\fastq{\gls{FASTQ}}
\newcommand*\mzml{mzML}

%%annotation
\newcommand*\hg[1]{\gls{GRCh}#1}
\newcommand*\ens[1]{ENSEMBL #1}


%abbr. for people
\newcommand*\alvis{Dr Alvis Brazma}
\newcommand*\nuno{Dr Nuno Fonseca}
\newcommand*\james{Dr James Wright}
\newcommand*\jyoti{Dr Jyoti Choudhary}
\newcommand*\mar{Dr Mar Gonzales-Porta}
\newcommand*\angela{Dr Angela Gonzales}
\newcommand*\johan{Dr Johan Rung}
\newcommand*\sarah{Dr Sarah Teichmann}
\newcommand*\gos{Dr Gos Micklem}
\newcommand*\wolfgang{Dr Wolfgang Huber}

%other abbr related to this project
\newcommand*\derivativeWork{\minisec{Communication to the community derived from this chapter}}
\newcommand*\pc{protein-coding}
\newcommand*\pcg{\pc\ gene}
\newcommand*\pcgs{\pcg{}s}
\newcommand*\cv{correlation of variation}
\newcommand*\cvs{correlations of variation}
\newcommand*\ttest{\textit{t}-test}
\newcommand*\pvalue[1]{\textit{p}-value ${#1}$}
\newcommand*\Welchttest{Welch's Two Sample \ttest}
\newcommand*\studenttest{Student's Two Sample \ttest}


\newcommand*\setOne{$\mathcal{W}_1$}
\newcommand*\setTwo{$\mathcal{W}_2$}

%% an easier life
\newcommand*\Href[1]{\href{#1}{#1}}
\newcommand*\mycite[1]{$\lbrack$\cite{#1}$\rbrack$}
\newcommand*\Paper[1]{\paper{\citetitle{#1}}}
\newcommand*\subminisec[1]{\quad\textbullet~\textbf{\small #1}\\[9pt]}
\newcommand*\subsubminisec[1]{\qquad\rightarrow~\textit{\small #1}\\[9pt]}
\newcommand*\hFo[2]{\href{#2}{#1}\footnote{{#1} --- \href{#2}{#2}}}
\newcommand*\hFoCi[3]{\hFo{#1}{#2}~\mycite{#3}}

\newcommand*\softFo[2]{\href{#2}{\soft{#1}}\footnote{\soft{#1} --- \href{#2}{#2}}}
\newcommand*\softCi[2]{\soft{#1}~\mycite{#2}}
\newcommand*\softFoCi[3]{\href{#2}{\soft{#1}}\footnote{\soft{#1} ---
\href{#2}{#2}}~\mycite{#3}}

\newcommand*\softFull[4]{\href{#2}{\soft{#1}}\footnote{\soft{#1} ---
\Href{#2}} ({#4})~\mycite{#3}}

\newcommand*\crefp[2]{\cref{#1}#2 (\vpageref{#1})}
\newcommand*\Crefp[2]{\Cref{#1}#2 (\vpageref{#1})}

%url to different part of the github repo for this thesis
\newcommand*\addressToirapConfFiles{%
\href{https://github.com/barzine/BaselineAtlas/tree/master/data/irap/config/human}%
{my personal Github repository}\footnote{%
\href{https://github.com/barzine/BaselineAtlas/tree/master/data/irap/config/human}%
{https://github.com/barzine/BaselineAtlas/tree/master/data/irap/config/human}}}


%abbr. for the tissues.

\newcommand*\Heart{\tissue{Heart}}
\newcommand*\Kidney{\tissue{Kidney}}
\newcommand*\Liver{\tissue{Liver}}
\newcommand*\Testis{\tissue{Testis}}

\newcommand*\Adipose{\tissue{Adipose}}
\newcommand*\Adrenal{\tissue{Adrenal gland}}
\newcommand*\Urinarybladder{\tissue{Bladder}}
\newcommand*\Bladder{\tissue{Bladder}}
\newcommand*\Cortex{\tissue{Cerebral cortex}}
\newcommand*\hColon{\tissue{Colon}}
\newcommand*\Lung{\tissue{Lung}}
\newcommand*\Esophagus{\tissue{Oesophagus}}
\newcommand*\Oesophagus{\tissue{Oesophagus}}
\newcommand*\Fallopian{\tissue{Fallopian tube}}
\newcommand*\Ovary{\tissue{Ovary}}
\newcommand*\Pancreas{\tissue{Pancreas}}
\newcommand*\Prostate{\tissue{Prostate}}
\newcommand*\Salivary{\tissue{Salivary gland}}
\newcommand*\Skeletal{\tissue{Skeletal muscle}}
\newcommand*\Skin{\tissue{Skin}}
\newcommand*\Intestine{\tissue{Small intestine}}
\newcommand*\Spleen{\tissue{Spleen}}
\newcommand*\Stomach{\tissue{Stomach}}
\newcommand*\Thyroid{\tissue{Thyroid}}
\newcommand*\Uterus{\tissue{Uterus}}

\newcommand*\heart{\Heart}
\newcommand*\kidney{\Kidney}
\newcommand*\liver{\Liver}
\newcommand*\testis{\Testis}

\newcommand*\adipose{\Adipose}
\newcommand*\adrenal{\Adrenal}
\newcommand*\urinarybladder{\Bladder}
\newcommand*\bladder{\Bladder}
\newcommand*\cortex{\cortex}
\newcommand*\hcolon{\hColon}
\newcommand*\lung{\Lung}
\newcommand*\esophagus{\Oesophagus}
\newcommand*\oesophagus{\Oesophagus}
\newcommand*\fallopian{\Fallopian}
\newcommand*\ovary{\Ovary}
\newcommand*\pancreas{\Pancreas}
\newcommand*\prostate{\Prostate}
\newcommand*\salivary{\Salivary}
\newcommand*\skeletal{\Skeletal}
\newcommand*\skin{\Skin}
\newcommand*\intestine{\Intestine}
\newcommand*\spleen{\Spleen}
\newcommand*\stomach{\Stomach}
\newcommand*\thyroid{\Thyroid}
\newcommand*\uterus{\Uterus}





\newcommand*\subcap[1]{\small{#1}}


%%% TOC customisation
\setcounter{secnumdepth}{5}
\setcounter{tocdepth}{7}


%\titleformat*{\section}{\LARGE\bfseries}
%\titleformat*{\subsection}{\Large\bfseries}
%\titleformat*{\subsubsection}{\large\bfseries}
%\titleformat*{\paragraph}{\large\bfseries}
%\titleformat*{\subparagraph}{\large\bfseries}


