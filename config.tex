\usepackage{scrhack}
\usepackage[dottedtoc,eulerchapternumbers]{classicthesis}
%\usepackage{arsclassica}
\usepackage{tocbasic}
\usepackage{polyglossia}
\setdefaultlanguage[variant=uk]{english}
%\setmainlanguage[variant=uk]{english}
\usepackage[english, iso]{isodate}
\usepackage{hyperref}
\usepackage[natbib=true,backend=biber,style=authoryear,maxcitenames=1,maxbibnames=99,useprefix,backref]{biblatex}
\usepackage{minted}
%\usepackage[explicit]{titlesec}
\usepackage{csquotes}
\usepackage{epigraph}
\usepackage{booktabs,dcolumn}
\usepackage{rotating}
\usepackage{multirow}
%\usepackage[table,xcdraw]{xcolor}
\usepackage{relsize}
\usepackage[onehalfspacing]{setspace}
%\usepackage{setspace}
%\usepackage[multidot]{grffile}
\usepackage{marginnote}
\usepackage{float}
\usepackage{fontspec}
\usepackage{comment}
\usepackage{amsmath}
\usepackage{unicode-math}
\usepackage{xpatch}
\usepackage{xltxtra}
\usepackage{xcolor}
\usepackage{calc}
\usepackage{ragged2e}
\usepackage[font={small},format=hang,labelfont={bf,sf},textfont=normal,up,margin=5pt,labelformat=default,labelsep=period]{caption}
%format=hang
%\usepackage[font=small,format=plain,labelfont=bf,up,textfont=normal,up,justification=justified,singlelinecheck=false,margin=5pt]{caption}
\usepackage{subcaption}
\usepackage{xr} %for cross reference between files

%\usepackage{ifoddpage}
\usepackage{afterpage}
\usepackage{verbatim}
%\usepackage{xspace} %the original author doesn't recommend it

\usepackage{pgfplots}
\pgfplotsset{compat=1.13}
\usepackage{expdlist}
\usepackage{pdflscape}
\usepackage{longtable}

\usepackage{tracklang}
\usepackage{mfirstuc}
\usepackage{etoolbox}
\usepackage{xkeyval}
\usepackage{textcase}
\usepackage{xfor}
\usepackage{datatool-base}
\usepackage{amsgen}
\usepackage{ragged2e}

\usepackage%[style=long,nolist] %as advised in classicthesis man
[xindy, toc, nopostdot,style=long,nolist]
{glossaries} %%hyperref has to be already loaded
\usepackage[automake,nomain,acronym]{glossaries-extra}
\setglossarystyle{longheader}
%\setabbreviationstyle[acronym]{long-short}
\makeglossaries%
%\loadglsentries{glossary.tex}

\usepackage{bookmark}
\usepackage{geometry}%% for page layout
\usepackage[nameinlink, noabbrev]{cleveref}


\setmainfont[Mapping=tex-text, Numbers=OldStyle]{TeX Gyre Pagella}
%\setsansfont[Mapping=tex-text, Numbers=OldStyle]{Droid Sans}
%\setmonofont{Droid Sans Mono}


\setmainfont[Ligatures=TeX, Numbers=OldStyle]{TeX Gyre Pagella}
%%\setmainfont[Ligatures=TeX, Numbers=OldStyle]{Helvetica}

\setsansfont[
    Ligatures=TeX,
    UprightFont={* Light},
    BoldFont={*},
    ItalicFont={* Light Italic},
    BoldItalicFont={* Italic},
    Scale=MatchLowercase
]{Gill Sans}
\setmonofont[Scale=MatchLowercase]{WenQuanYi Zen Hei Mono}
\setmathfont[
    Extension=.otf,
    BoldFont=*bold,
]{xits-math}


%%%% Check with classicthesis manual to redifine the chapter and sections formats

%\titleformat{\chapter}[display]%
%{\relax}{\mbox{}\oldmarginpar{\vspace*{-3\baselineskip}\color{halfgray}\chapterNumber\thechapter}}{0pt}%
%{\raggedright\spacedallcaps}[\Huge\vspace*{.8\baselineskip}\titlerule]%
% sections
%\titleformat{\section}
%{\relax}{\textbf{\MakeTextLowercase{\thesection}}}{1em}{\spacedlowsmallcaps}
% subsections
%\titleformat{\subsection}{\relax}{\textsc{\MakeTextLowercase{\thesubsection}}}{1em}{\normalsize}



%\setkomafont{title}{\rmfamily\Huge}
%\setkomafont{subject}{\normalfont\normalcolor}
\isodash{\ensuremath{\cdot}}% Replacement for `‧` = U+‧2027 HYPHENATION POINT

%\definecolor{cambridgeblue}{RGB}{163, 193, 173}
%\definecolor{cambridgebluedark}{RGB}{17, 94, 103}


%\defaultfontfeatures{Ligatures=TeX,Numbers=OldStyle}

\addbibresource{Bibliography.bib}

\setstretch{1.1}
\graphicspath{{gfx/}} %as instructed in classicthesis

% For debugging: showframe=true to see the layout frames
\geometry{bottom=65mm,showframe=false,margin=35mm,marginparsep=3mm,marginparwidth=20mm}




\pagestyle{scrheadings}


%%%%%%%
%%% customisation stollen from Manuel Kuehner template
%%some calculations
%% calc package is needed which is loaded here: 01_Preamble/CommonPackages.tex
%% If you want to understand the calculations visit:
%% http://en.wikibooks.org/wiki/LaTeX/Page_Layout
\newlength{\myLenghthFootAbstand}
\setlength{\myLenghthFootAbstand}{\paperheight-1in-\topmargin- \headheight-\headsep-\textheight-\footskip}
\newlength{\myLenghthTemp}
\setlength{\myLenghthTemp}{\myLenghthFootAbstand+\baselineskip}

\clearscrheadfoot%
% Header
\ohead{%
    \headmark%
    }
% Left (even page numbers) footer
\lefoot%
[% scrplain style (begin)
    \setlength{\unitlength}{\myLenghthFootAbstand}%
    \begin{picture}(0,0)%
        \put(0,-1)%
        {%
            \makebox(0,0)[lb]%
            {%
                \rule{0.4pt}{\myLenghthTemp}%
            }%
        }%
    \end{picture}
    \llap{\pagemark~}
]% scrplain style (end)
%
{% scrheadings style (begin)
    \setlength{\unitlength}{\myLenghthFootAbstand}%
    \begin{picture}(0,0)%
        \put(0,-1)%
        {%
            \makebox(0,0)[lb]%
            {%
                \rule{0.4pt}{\myLenghthTemp}%
            }%
        }%
    \end{picture}
    \llap{\pagemark~}%
    }% scrheadings style (end)

%% Right (odd page numbers) footer
\rofoot%
[% scrplain style (begin)
    \rlap{~\pagemark}%%
    \setlength{\unitlength}{\myLenghthFootAbstand}%
    \begin{picture}(0,0)%
        \put(0,-1)%
        {%
            \makebox(0,0)[lb]%
            {%
                \rule{0.4pt}{\myLenghthTemp}%
            }%
        }%
    \end{picture}%
    ]% scrplain style (end)
    %
{% scrplain style (begin)
    \rlap{~\pagemark}%%
    \setlength{\unitlength}{\myLenghthFootAbstand}%
    \begin{picture}(0,0)%
        \put(0,-1)%
        {%
            \makebox(0,0)[lb]%
            {%
                \rule{0.4pt}{\myLenghthTemp}%
            }%
        }%
    \end{picture}%
    }% scrplain style (end)





\setlength{\headheight}{1.1\baselineskip}
%%%%%%%


% %%% More layout definitions
%    \definecolor{primary}{HTML}{429EB6}
%    \definecolor{secondary}{HTML}{DE8950}
%    \definecolor{tertiary}{HTML}{4DB966}
%    \definecolor{quarternary}{HTML}{F3B5FB}
%    \definecolor{quinary}{HTML}{E7BE05}
%    \newcommand*\primaryname{blue}
%    \newcommand*\secondaryname{orange}
%    \newcommand*\tertiaryname{green}
%    \newcommand*\quarternaryname{purple}
%    \newcommand*\quinaryname{yellow}
%
%    \colorlet{thesis@toccolor}{black}
%    \colorlet{thesis@linkcolor}{black!70}
%
%    \definecolor{codenormal}{HTML}{404040}
%    \AtBeginEnvironment{minted}{\color{codenormal}}
%    \usemintedstyle{klmrthesis}

%   \hypersetup{%
%        hypertexnames=false,
%        linktoc=all,
%        colorlinks,
%        citecolor=thesis@linkcolor,
%        linkcolor=thesis@linkcolor,
%        filecolor=thesis@linkcolor,
%        urlcolor=thesis@linkcolor
%    }


%%%%%%%
%%%% custom macros
\newcommand*\captitle[1]{\textbf{#1}}
\setkomafont{caption}{\bfseries\sffamily}

%if use of classicthesis
\newcommand*\todo[1]{%
    \graffito{\textcolor{red}{TO\ DO:~#1}}}

%if no use of classicthesis but \usepackage{scrpage2} instead
%\newcommand*\todo[1]{%
%    \comment{TO\ DO:#1}}

\newcommand{\fixme}[1]{{\let\marginpar\oldmarginpar\todo{#1}}}

\newcommand*\TK[1]{
    \textcolor{blue}{[TK:\ #1]}}

\newcommand*\Rough[1]{
    Speak about:
    \textsl{\textcolor{gray}{#1}}}

\newcommand*\rough[1]{\textsl{\textcolor{gray}{#1}}}

%%%% shortcuts
\newcommand*\myPlace{Cambridge (UK)}
\newcommand*\myDate{7 September 2016}

\newcommand*\mol[1]{\textnormal{#1}}
\newcommand*\gene[1]{\textit{#1}}
\newcommand*\ko[1]{\textit{#1\textsuperscript{\(-/-\)}}}
\newcommand*\protein[1]{#1}
\newcommand*\species[1]{\textit{#1}}

%\newcommand*\mRNA{\mol{mRNA}}
%\newcommand*\mRNAs{\mol{mRNAs}}
\newcommand*\mRNA{\gls{mRNA}}
\newcommand*\mRNAs{\glspl{mRNA}}

\newcommand*\tissue[1]{\emph{#1}}

%\newcommand*\DNA{\mol{DNA}}
\newcommand*\DNA{\gls{DNA}}

\newcommand*\dataset[1]{\textnormal{#1}}
\newcommand*\etal{\textit{et al.}}
\newcommand*\Rnaseq{\gls{RNA-Seq}}

\newcommand*\paper[1]{\emph{#1}}
\newcommand*\subcap[1]{\small{#1}}

\newcommand*\myName{Mitra Parissa Barzine}

%%% TOC customisation
\setcounter{secnumdepth}{5}
\setcounter{tocdepth}{7}


%\titleformat*{\section}{\LARGE\bfseries}
%\titleformat*{\subsection}{\Large\bfseries}
%\titleformat*{\subsubsection}{\large\bfseries}
%\titleformat*{\paragraph}{\large\bfseries}
%\titleformat*{\subparagraph}{\large\bfseries}


