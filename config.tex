\usepackage{scrhack}
\usepackage[dottedtoc,eulerchapternumbers]{classicthesis}
%\usepackage{arsclassica}
\usepackage{tocbasic}
\usepackage{polyglossia}
\usepackage[english, iso]{isodate}
\usepackage{csquotes}
\usepackage{epigraph}
\usepackage{hyperref}
\usepackage[natbib=true,backend=biber,style=authoryear,maxcitenames=1,maxbibnames=99,useprefix]{biblatex}
\usepackage{booktabs}
\usepackage{relsize}
\usepackage{setspace}
%\usepackage[multidot]{grffile}
\usepackage{fontspec}
\usepackage{comment}

\usepackage{amsmath}
\usepackage{unicode-math}
\usepackage{minted}
\usepackage{xpatch}
\usepackage{xltxtra}
\usepackage{xcolor}
\usepackage{calc}
\usepackage{ragged2e}
\usepackage[font={small},labelfont={bf,sf},format=hang,margin=5pt]{caption}
\usepackage{verbatim}
%% \usepackage{xspace} %the original author doesn't recommend it


\usepackage{pgfplots}
\pgfplotsset{compat=1.13}


\usepackage{tracklang}
\usepackage{mfirstuc}
\usepackage{etoolbox}
\usepackage{xkeyval}
\usepackage{textcase}
\usepackage{xfor}
\usepackage{datatool-base}
\usepackage{amsgen}


\usepackage%[style=long,nolist] %as advised in classicthesis man
[xindy, toc, nopostdot,style=long,nolist]
{glossaries} %%hyperref has to be already loaded
\usepackage[automake,nomain,acronym]{glossaries-extra}
%\setabbreviationstyle[acronym]{long-short}
\makeglossaries%
%\loadglsentries{glossary.tex}


\usepackage[nameinlink, noabbrev]{cleveref}


\setmainlanguage[variant=uk]{english}
%\setdefaultlanguage[variant=uk]{english}
%\setmainfont[Mapping=tex-text, Numbers=OldStyle]{TeX Gyre Pagella}
%\setsansfont[Mapping=tex-text, Numbers=OldStyle]{Droid Sans}
%\setmonofont{Droid Sans Mono}


\setmainfont[Ligatures=TeX, Numbers=OldStyle]{TeX Gyre Pagella}
%%\setmainfont[Ligatures=TeX, Numbers=OldStyle]{Helvetica}

\setsansfont[
    Ligatures=TeX,
    UprightFont={* Light},
    BoldFont={*},
    ItalicFont={* Light Italic},
    BoldItalicFont={* Italic},
    Scale=MatchLowercase
]{Gill Sans}
\setmonofont[Scale=MatchLowercase]{WenQuanYi Zen Hei Mono}
\setmathfont[
    Extension=.otf,
    BoldFont=*bold,
]{xits-math}


\setkomafont{title}{\rmfamily\Huge}
\setkomafont{subject}{\normalfont\normalcolor}
\isodash{\ensuremath{\cdot}}% Replacement for `‧` = U+‧2027 HYPHENATION POINT

\definecolor{cambridgeblue}{RGB}{163, 193, 173}
\definecolor{cambridgebluedark}{RGB}{17, 94, 103}


%\defaultfontfeatures{Ligatures=TeX,Numbers=OldStyle}

\addbibresource{Bibliography.bib}

\setstretch{1.1}
\graphicspath{{gfx/}} %as instructed in classicthesis

\usepackage{geometry}%% for page layout
% For debugging: showframe=true to see the layout frames
\geometry{bottom=65mm,showframe=false,margin=35mm,marginparsep=3mm,marginparwidth=20mm}




\pagestyle{scrheadings}

%%%%%%
%% customisation stollen from Manuel Kuehner template
%some calculations
% calc package is needed which is loaded here: 01_Preamble/CommonPackages.tex
% If you want to understand the calculations visit:
% http://en.wikibooks.org/wiki/LaTeX/Page_Layout
\newlength{\myLenghthFootAbstand}
\setlength{\myLenghthFootAbstand}{\paperheight-1in-\topmargin- \headheight-\headsep-\textheight-\footskip}
\newlength{\myLenghthTemp}
\setlength{\myLenghthTemp}{\myLenghthFootAbstand+\baselineskip}

\clearscrheadfoot%
% Header
\ohead{%
    \headmark%
    }
% Left (even page numbers) footer
\lefoot%
[% scrplain style (begin)
    \setlength{\unitlength}{\myLenghthFootAbstand}%
    \begin{picture}(0,0)%
        \put(0,-1)%
        {%
            \makebox(0,0)[lb]%
            {%
                \rule{0.4pt}{\myLenghthTemp}%
            }%
        }%
    \end{picture}\llap{\pagemark~}
]% scrplain style (end)
%
{% scrheadings style (begin)
    \setlength{\unitlength}{\myLenghthFootAbstand}%
    \begin{picture}(0,0)%
        \put(0,-1)%
        {%
            \makebox(0,0)[lb]%
            {%
                \rule{0.4pt}{\myLenghthTemp}%
            }%
        }%
    \end{picture}\llap{\pagemark~}%
    }% scrheadings style (end)

% Right (odd page numbers) footer
\rofoot%
[% scrplain style (begin)
    \rlap{~\pagemark}%%
    \setlength{\unitlength}{\myLenghthFootAbstand}%
    \begin{picture}(0,0)%
        \put(0,-1)%
        {%
            \makebox(0,0)[lb]%
            {%
                \rule{0.4pt}{\myLenghthTemp}%
            }%
        }%
    \end{picture}%
    ]% scrplain style (end)
    %
{% scrplain style (begin)
    \rlap{~\pagemark}%%
    \setlength{\unitlength}{\myLenghthFootAbstand}%
    \begin{picture}(0,0)%
        \put(0,-1)%
        {%
            \makebox(0,0)[lb]%
            {%
                \rule{0.4pt}{\myLenghthTemp}%
            }%
        }%
    \end{picture}%
    }% scrplain style (end)

%%%%%%%


%%% More layout definitions
    \definecolor{primary}{HTML}{429EB6}
    \definecolor{secondary}{HTML}{DE8950}
    \definecolor{tertiary}{HTML}{4DB966}
    \definecolor{quarternary}{HTML}{F3B5FB}
    \definecolor{quinary}{HTML}{E7BE05}
    \newcommand*\primaryname{blue}
    \newcommand*\secondaryname{orange}
    \newcommand*\tertiaryname{green}
    \newcommand*\quarternaryname{purple}
    \newcommand*\quinaryname{yellow}

    \colorlet{thesis@toccolor}{black}
    \colorlet{thesis@linkcolor}{black!70}

    \definecolor{codenormal}{HTML}{404040}
    \AtBeginEnvironment{minted}{\color{codenormal}}
    \usemintedstyle{klmrthesis}

    \hypersetup{%
        hypertexnames=false,
        linktoc=all,
        colorlinks,
        citecolor=thesis@linkcolor,
        linkcolor=thesis@linkcolor,
        filecolor=thesis@linkcolor,
        urlcolor=thesis@linkcolor
    }


%%%%%%%
%%%% custom macros
\newcommand*\captitle[1]{\textbf{#1}}


%if use of classicthesis
\newcommand*\todo[1]{%
    \graffito{\textcolor{red}{TO\ DO:#1}}}

%if no use of classicthesis but \usepackage{scrpage2} instead
%\newcommand*\todo[1]{%
%    \comment{TO\ DO:#1}}

\newcommand{\fixme}[1]{{\let\marginpar\oldmarginpar\todo{#1}}}

\newcommand*\TK[1]{
    \Large{\textcolor{blue}{[TK:\ #1]}}}

\newcommand*\Rough[1]{
    Speak about:

    \textsl{\textcolor{gray}{#1}}}

%%%% shortcuts
\newcommand*\myPlace{Cambridge (UK)}
\newcommand*\myDate{28 July 2016}

\newcommand*\mol[1]{\textnormal{#1}}
\newcommand*\gene[1]{\textit{#1}}
\newcommand*\ko[1]{\textit{#1\textsuperscript{\(-/-\)}}}
\newcommand*\protein[1]{#1}
\newcommand*\species[1]{\textit{#1}}

\newcommand*\mRNA{\mol{mRNA}}
\newcommand*\DNA{\mol{DNA}}
\newcommand*\dataset[1]{\textnormal{#1}}



\newcommand*\myName{Mitra Parissa Barzine}

%%% TOC customisation
\setcounter{secnumdepth}{5}
\setcounter{tocdepth}{7}

