\chapter{Introduction}
\label{ch:Introduction}

\begin{comment}
\setlength{\epigraphwidth}{0.6\textwidth}
\setlength{\epigraphrule}{0.1pt}
%\epigraphhead[70]{%
    \epigraph{It would probably be oversimplifying the
matter,\\but I am strongly tempted to say,\\ ``All life is nucleic acid; the rest is
commentary''.}{\cite{asimov:WrongRel}}%
%}
\end{comment}

\begin{comment}
When Asimov concluded with these words his chapter ``Beginning with bone'',
scientists had already started unfolding one of the most mesmerising biological
mystery: how does the Nature manage and organise the production of specific
effectors in specific locations. In other words how from one cell
\TK{revoir cette phrase: idea initiale pourquoi
organisme A different
de B, ou un foie et un foie et pas un coeur + comment d'une cellule-oeuf on a un
organisme complet qui se creée... ne pas s'étendre dessus par contre}.\
Watson and Crick by publishing the double-helix structure of \DNA\
\citep{DNA1953} unlocked our path to understand the natural ways of storing and
manipulating the information. As the whole past six decades were packed with major
discoveries and technological achievements, there are many, fascinating,
milestones that could be discussed.
The following pages present a summary of the facts and techniques
that form the biological and technical context of the work backing this thesis.

--- ici intro par rapport à Asimov et le fait que l'ADN est un 'blueprint' et qu'en
fonction du l'organe ou du develepment stage l'expression est régulé. On sait
maitenant que la régulation se passe à différent niveau.... finir peut être la
section sur Harvey qui a élucider la circulation sanguine car il a utilisé
l'observation et quantification du sang ---


As where the situation stands for now, we know that between the unique common
bluprint, which is the \DNA\ and the different phenotypes displayed by the tissues
comprising our bodies, they are many regulatory processes happening at different
layers (transcription, and translation).

 ### agent


\Rough{%
Gene expression, transcription, translation, regulation at each stage,
expected and reported correlation between transcripts and proteins
Measuring transcript expression by RNAseq, wet-lab part, analysis methods,
available large scale datasets on human tissues
Measuring protein expression by MS, data analysis – spectral counts, top 3, etc,
available large scale datasets on human tissues
The problem of comparing and integrating independent datasets, EBI’s GXA \\ \\
key concepts: DNA storage, RNA transfert of information and regulation,
protein effectors ==\textgreater phenotype.\\ \\
%
Do not forget the (tissues) specifications and the structural functions. =\textgreater Embryology.\\ \\
%
introduction real possible start: la soupe primaire -\textgreater creation de RNA, aa,
oeuf/poule
premiere cellule, \ldots (lire Darwin, il y a ptet moyen d ajouter une citation ou
quelque chose.%
}

\begin{itemize}
    \item Proteomics messy and noisy, challenging \ldots
    \item Transcriptomics seem a good proxy to study the phenotype => one kind of molecule made of the same pieces.
    \item Technology improved a lot from the microarrays to sequencing.
\end{itemize}

\end{comment}


\section{Diversity and universality of Life}



%\TK{Soupe primaire experience du Chimiste}
\TK{The central dogma of Molecular biology} => Crick describes the cycle

\fixme{Add references on replications, not going into that one}

\subsection{Translation + Regulation}

\subsection{Translation + Regulation}

Just so that the next part of the chapter got a OK presentation:
\begin{itemize}
    \item \gls{mRNA}
    \item \gls{ncRNA}
    \item \gls{tRNA}
    \item \gls{miRNA}
    \item \gls{snRNA}
    \item \gls{scaRNA}
    \item \gls{snoRNA}
    \item \gls{scRNA}
    \item \gls{rRNA}
\end{itemize}

\TK{Assumption mRNA and proteins : highly correlated [find references]}


\section{Capturing the expression in the laboratory}

Biological research uses mainly two approaches to
study the cell life intricacy and its underlying mechanisms.\\
The oldest approach is descriptive:


Small bits: corrections for Rnaseq are already part of the analysis and
requires often as much flair than skills.
Contrarily to \Dnaseq\ where corrections can be applied \TK{add reference} and
then the analysis be done, in Rnaseq each analysis requires a set of conform
quantification and normalisation methods.
While, there are quite established protocols for differential expression analysis,
there are presently many other downstream analyses that are cumbersome
and/or not settled yet. This is the case for this study.


Corrections for Rnaseq are already part of the analysis and
requires often as much flair than skills.

The transcriptome is the total repertoire of transcripts (\ie\ \glspl{RNA}
molecules) expressed in a cell or tissue at a given time and condition. Unlike
the genome which is roughly identical regardless which cell of a particular
individual is considered, the transcriptome varies\ldots

\subsection{Transcriptome exploration with RNA sequencing}

\TK{history of \Rnaseq.}
In the past decade, \gls{RNA-Seq} technology has risen as the method of choice
for  transcriptome.

Many methods and technologies through the years but more recently, boom of study:
next generation sequencing (1st generation, 2nd generation and 3rd)\ldots So
much that now the expression doesn't mean anything.

Completion of the Human genome project : key changer: probes with microarrays
possible (as there were then template). Next key changer: shotgun sequencing
instead of Sanger sequencing (slow). + advance in computer science: needs of
parallelisation and storage.

So, In 2008, shift from microarrays to \Rnaseq.

\rew{
The first step in transcriptome sequencing is library preparation, and consists of obtaining the starting material and converting it into a cDNA library that can be loaded into the sequencing machine (Figure 1.5) [van Dijk et al., 2014]. Following RNA extraction, the RNA species of interest are typically enriched through either polyA selection or ribodepletion. In both cases, the aim is to diminish the concentration of rRNAs, i.e. the most abundant species of RNA in the cell. With the first method, this is achieved through the use of oligo-dT beads, which enable the specific extraction of polyAdenylated RNAs, hence ensuring a good representation of mRNAs (Figure 1.5 - step 1). Conversely, ribodepletion relies on the use of ribonucleases to specifically digest rRNAs, and has the advantage of not restricting the analyses to a specific type of RNA. Indeed, the term total RNA is typically used to refer to datasets produced with such protocol, while those obtained with the former method are commonly known as polyA-selected. Due to the simpler protocol and its lower price, polyA selection emerges as the most popular choice amongst the currently available RNA-seq datasets, with the notable exception of those studies aimed at characterising non-coding RNA species, which typically lack a polyA tail. The extracted RNA is then fragmented via hydrolysis with divalent cations and retro-transcribed into double stranded cDNA by using random hexamer primers, since the sequence of the obtained fragments is not known at this point (Figure 1.5 - step 2). These steps are followed by the ligation of adapter sequences at both ends of each cDNA fragment (Figure 1.5 - step 3). Such adapters satisfy two different purposes: on the one hand, they enable the hybridisation of those fragments into the flow cell,
where the sequencing takes place; on the other hand, they serve as primers for the sequencing reaction. Then, the resulting cDNA fragments are size-selected through gel electrophoresis to fit within the range required by the sequencing machine (typically 300-500 bp). Fragments outside this range will be missed; hence the existence of alternative protocols for the study of small RNAs [Zhuang et al., 2012]. Finally, the cDNA library is amplified by PCR.
}


In the following section, I introduce the typical steps of the required workflow to
study the transcriptome through sequencing on an Illumina platform. In fact,
while not by prior design, all the transcriptomes analysed in this thesis are the
product of Illumina sequencing \footnote{While there are many types of sequencing
platform, Illumina is still by far the most popular: on 2,851
\species{Homo sapiens} \Rnaseq\ studies included in ArrayExpress, were sequenced
with an Illumina sequencing
machine (query done on 14 June 2017)}.
Other platforms protocols will need
various modifications that are outside of the realm of this thesis and that are
well explained in the respective manual of each sequencer machine.

I will discuss the concepts behind this type of sequencing technology and the
methods for analysing this kind of data with emphasis on the tools and
approaches I used.


\subsubsection{Quality assessment, trimming and filtering}





\NB Generally, after the sequencer calls the reads, a first trimming removes
all the adaptors and barcodes needed by the sequencing protocols. Thus,
in principle, they are not to be found in raw data from repositories.
However, to avert any latter contingency, a research against
a list of the most common adaptors and an over-representation assessment of small
sequences (\emph{k-mers}) at each end of the reads is good practice.




The quality assessment allows to remove any read (or part of it) that would
increase the complexity of the mapping step or skew the downstream analyses.

Whenever the overall quality of the samples allows it, it is best to discard very


Then, reads that present an overall quality score below a given threshold (10) are
fully discarded. Reads that have uncalled bases (\textsc{N}) are also discarded.
Since the quality decreases as the calling process progresses, all the reads are
trimmed to a same length\footnote{This is a requirement of some tools (mappers
in particular).} in a way to optimise the purity-length balance. The trimming has
to be less than 15\% of the original length. If needed, more reads are discarded
so the length is maximised.








\subsection{Transcriptomics Studies}
    \subsubsection{Main technologies}
While microarrays are measuring many \mRNAs\ at once, their number is limited,
    \subsubsection{A typical sequencing workflow}
\subsection{Proteomics}
    \subsubsection{Main technologies}
    \subsubsection{Mass sprectrometry}

\section{Big studies, big data and Reproducibility}



\begin{comment}
\subsection{Reproducibility issues}
        \begin{itemize}
            \item{Different samples}
            \item{Technology: wet lab but also software: Rupgrade, ....}
            \item{missing meta-data}
        \end{itemize}
    \subsection{Main concerns}
        \subsubsection{Detection}
        \subsubsection{Quantification}
    \subsection{Consistency through biological layers}
\end{comment}


\subsection{Co-studies on Transcriptomics and Proteomics in the literature}
\subsection{EBI Expression Atlas or how integrating independent datasets }


\section{Aims of the thesis}

One main focus of my work for this thesis was to appraise the
consistency of findings that have been enlightened in normal Human tissues by
individual large scale transcriptomics and, more recently, large-scale
proteomics studies.

While the amount of data to process and integrate can be challenging,
paradoxically the data available to draw statistically significant conclusions
is often still too sparse. However, we do have enough to provide at least a
qualitative assessment of the consistency and evaluate the feasibility of a
quantitative assessment in general, if not, realise a quantitative integration
for a few key tissues.

\section{Achievements}
\begin{itemize}
    \item First time so many datasets have been reprocessed together with the
        same annotation and hence comparison are better (y'a le truc de chez Burges,
        mais ils se sont concentré sur une petite partie des genes)
    \item surtout comparison of Gtex and Uhlen (aussi approfondi)
    \item reprocessing of all the proteome untargeted in human
    \item comparison entre protein expression and 2 datasets (meilleur que le
        science report dans le sens où il y a pas de polémique
    \item siteweb//application de visualisaiton de la comparison
    \item liste croisé de tissues specifique et house keeping à travers datasets
        transcriptomic et protoemic
    \item attempt de normalisation basée sur les housekeepings
\end{itemize}


\section{Reproducibility vs. Repeatability}

\section{Comment preparer RNA-seq libraries?}

\section{Paired-end, Single-end}

\section{Experimental design}
\subsection{Technical replicates}
\subsection{Biological replicates}
pourquoi c'est important.
