\chapter{Introduction}

\epigraph{All life is nucleic acid; the rest is commentary.}{I. Asimov -
"The Relativity of Wrong" (1988) - "Beginning with Bone" (May 1987)}


\Rough{
Gene expression, transcription, translation, regulation at each stage, expected and reported correlation between transcripts and proteins
Measuring transcript expression by RNAseq, wet-lab part, analysis methods, available large scale datasets on human tissues
Measuring protein expression by MS, data analysis – spectral counts, top 3, etc, available large scale datasets on human tissues
The problem of comparing and integrating independent datasets, EBI’s GXA

key concepts: DNA storage, RNA transfert of information and regulation, protein effectors ==> phenotype

}

    \begin{itemize}
        \item Proteomics messy and noisy, challenging \ldots
        \item Transcriptomics seem a good proxy to study the phenotype => one kind of molecule made of the same pieces.
        \item Technology improved a lot from the microarrays to sequencing.
    \end{itemize}

\section{Life and expression}

\TK{The central dogma of Molecular biology}

\fixme{Add references on replications, not going into that one}

\subsection{Translation + Regulation}

\subsection{Translation + Regulation}

\TK{Assumption mRNA and proteins : highly correlated [find references]}


\section{Capturing the expression}
\subsection{Transcriptomics}
\subsection{Proteomics}

\section{Problem of integrating independent datasets ====> GXA}

\subsection{Consistency issues}

\subsection{Reproducibility issues : i.e. R upgrading, blablabla}

