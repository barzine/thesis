\chapter{Introduction}
\label{ch:Introduction}
\setlength{\epigraphwidth}{0.6\textwidth}
\setlength{\epigraphrule}{0.1pt}
%\epigraphhead[70]{%
    \epigraph{It would probably be oversimplifying the
matter,\\but I am strongly tempted to say,\\ ``All life is nucleic acid; the rest is
commentary''.}{\cite{asimov:WrongRel}}%
%}

\begin{comment}
When Asimov concluded with these words his chapter ``Beginning with bone'',
scientists had already started unfolding one of the most mesmerising biological
mystery: how does the Nature manage and organise the production of specific
effectors in specific locations. In other words how from one cell
\TK{revoir cette phrase: idea initiale pourquoi
organisme A different
de B, ou un foie et un foie et pas un coeur + comment d'une cellule-oeuf on a un
organisme complet qui se creée... ne pas s'étendre dessus par contre}.\
Watson and Crick by publishing the double-helix structure of \DNA\
\citep{DNA1953} unlocked our path to understand the natural ways of storing and
manipulating the information. As the whole past six decades were packed with major
discoveries and technological achievements, there are many, fascinating,
milestones that could be discussed.
The following pages present a summary of the facts and techniques
that form the biological and technical context of the work backing this thesis.

--- ici intro par rapport à Asimov et le fait que l'ADN est un 'blueprint' et qu'en
fonction du l'organe ou du develepment stage l'expression est régulé. On sait
maitenant que la régulation se passe à différent niveau.... finir peut être la
section sur Harvey qui a élucider la circulation sanguine car il a utilisé
l'observation et quantification du sang ---


As where the situation stands for now, we know that between the unique common
bluprint, which is the \DNA\ and the different phenotypes displayed by the tissues
comprising our bodies, they are many regulatory processes happening at different
layers (transcription, and translation).

 ### agent


\Rough{%
Gene expression, transcription, translation, regulation at each stage,
expected and reported correlation between transcripts and proteins
Measuring transcript expression by RNAseq, wet-lab part, analysis methods,
available large scale datasets on human tissues
Measuring protein expression by MS, data analysis – spectral counts, top 3, etc,
available large scale datasets on human tissues
The problem of comparing and integrating independent datasets, EBI’s GXA \\ \\
key concepts: DNA storage, RNA transfert of information and regulation,
protein effectors ==\textgreater phenotype.\\ \\
%
Do not forget the (tissues) specifications and the structural functions. =\textgreater Embryology.\\ \\
%
introduction real possible start: la soupe primaire -\textgreater creation de RNA, aa,
oeuf/poule
premiere cellule, \ldots (lire Darwin, il y a ptet moyen d ajouter une citation ou
quelque chose.%
}

\begin{itemize}
    \item Proteomics messy and noisy, challenging \ldots
    \item Transcriptomics seem a good proxy to study the phenotype => one kind of molecule made of the same pieces.
    \item Technology improved a lot from the microarrays to sequencing.
\end{itemize}

\end{comment}


\section{Diversity and universality of Life}




%\TK{Soupe primaire experience du Chimiste}
\TK{The central dogma of Molecular biology} => Crick describes the cycle

\fixme{Add references on replications, not going into that one}

\subsection{Translation + Regulation}

\subsection{Translation + Regulation}

Just so that the next part of the chapter got a OK presentation:
\begin{itemize}
    \item \gls{mRNA}
    \item \gls{ncRNA}
    \item \gls{tRNA}
    \item \gls{miRNA}
    \item \gls{snRNA}
    \item \gls{scaRNA}
    \item \gls{snoRNA}
    \item \gls{scRNA}
    \item \gls{rRNA}
\end{itemize}

\TK{Assumption mRNA and proteins : highly correlated [find references]}


\section{Capturing the expression in the laboratory}

Biological research uses mainly two approaches to
study the cell life intricacy and its underlying mechanisms.\\
The oldest approach is descriptive:


Small bits: corrections for \Rnaseq\ are already part of the analysis and
requires often as much flair than skills.
Contrarily to \Dnaseq\ where corrections can be applied \TK{add reference} and
then the analysis be done, in \Rnaseq each analysis requires a set of conform
quantification and normalisation methods.
While, there are quite established protocols for differential expression analysis,
there are presently many other downstream analyses that are cumbersome
and/or not settled yet. This is the case for this study.


Corrections for \Rnaseq\ are already part of the analysis and
requires often as much flair than skills.


\subsection{Transcriptomics Studies}
    \subsubsection{Main technologies}
While microarrays are measuring many \mRNAs\ at once, their number is limited,
    \subsubsection{A typical sequencing workflow}
\subsection{Proteomics}
    \subsubsection{Main technologies}
    \subsubsection{Mass sprectrometry}

\section{Big studies, big data and Reproducibility}


\subsection{Reproducibility issues}
        \begin{itemize}
            \item{Different samples}
            \item{Technology: wet lab but also software: Rupgrade, ....}
            \item{missing meta-data}
        \end{itemize}
    \subsection{Main concerns}
        \subsubsection{Detection}
        \subsubsection{Quantification}
    \subsection{Consistency through biological layers}

\begin{comment}


\subsection{Co-studies on Transcriptomics and Proteomics in the literature}
\subsection{EBI Expression Atlas or how integrating independent datasets }


\section{Aims of the thesis}

One main focus of my work for this thesis was to appraise the
consistency of findings that have been enlightened in normal Human tissues by
individual large scale transcriptomics and, more recently, large-scale
proteomics studies.

While the amount of data to process and integrate can be challenging,
paradoxically the data available to draw statistically significant conclusions
is often still too sparse. However, we do have enough to provide at least a
qualitative assessment of the consistency and evaluate the feasibility of a
quantitative assessment in general, if not, realise a quantitative integration
for a few key tissues.









