\chapter{Introduction}
\label{ch:Introduction}
\setlength{\epigraphwidth}{0.6\textwidth}
\setlength{\epigraphrule}{0.1pt}
%\epigraphhead[70]{%
    \epigraph{It would probably be oversimplifying the
matter,\\but I am strongly tempted to say,\\ ``All life is nucleic acid; the rest is
commentary''.}{\cite{asimov:WrongRel}}%
%}



When Asimov has written this words, the double-helix structure of DNA had been
resolved, the experiment of Miller gave 
\textbf{HS:}One of the earliest --- if not \emph{the} earliest --- questions historically
documented is the origins of Life. While mythologies and religions theses recorded
answer  these questions as early as History was recorded%
\TK{add first explanation of the origin of life ever recorded}

While we can not explain how life get on Earth or what was the first organism,
we have a clearer understanding on how Life evolve.


\Rough{%
Gene expression, transcription, translation, regulation at each stage,
expected and reported correlation between transcripts and proteins
Measuring transcript expression by RNAseq, wet-lab part, analysis methods,
available large scale datasets on human tissues
Measuring protein expression by MS, data analysis – spectral counts, top 3, etc,
available large scale datasets on human tissues
The problem of comparing and integrating independent datasets, EBI’s GXA \\ \\
key concepts: DNA storage, RNA transfert of information and regulation,
protein effectors ==\textgreater phenotype.\\ \\
%
Do not forget the (tissues) specifications and the structural functions. =\textgreater Embryology.\\ \\
%
introduction real possible start: la soupe primaire -\textgreater creation de RNA, aa,
oeuf/poule
premiere cellule, \ldots (lire Darwin, il y a ptet moyen d ajouter une citation ou
quelque chose.%
}

\begin{itemize}
    \item Proteomics messy and noisy, challenging \ldots
    \item Transcriptomics seem a good proxy to study the phenotype => one kind of molecule made of the same pieces.
    \item Technology improved a lot from the microarrays to sequencing.
\end{itemize}

\section{Life and expression}

\TK{Soupe primaire experience du Chimiste}
\TK{The central dogma of Molecular biology} => Crick describes the cycle

\fixme{Add references on replications, not going into that one}

\subsection{Translation + Regulation}

\subsection{Translation + Regulation}

\TK{Assumption mRNA and proteins : highly correlated [find references]}


\section{Capturing the expression in the laboratory}
\subsection{Transcriptomics Studies}
    \subsubsection{Main technologies}
    \subsubsection{A typical sequencing workflow}
\subsection{Proteomics}
    \subsubsection{Main technologies}
    \subsubsection{Mass sprectrometry}
\subsection{Co-studies on Transcriptomics and Proteomics in the literature}

\section{Big studies, big data and Reproducibility}
    \subsection{Available big high-throughput studies in Human}
    \subsection{Reproducibility issues}
        \begin{itemize}
            \item{Different samples}
            \item{Technology: wet lab but also software: Rupgrade, ....}
            \item{missing meta-data}
        \end{itemize}
    \subsection{Main concerns}
        \subsubsection{Detection}
        \subsubsection{Quantification}
    \subsection{Consistency through biological layers}
    \subsection{EBI Expression Atlas or how integrating independent
    datasets }

\section{Aims of the thesis}

TBC


Just so that the next chapter got a OK presentation:
\begin{itemize}
    \item \gls{mRNA}
    \item \gls{ncRNA}
    \item \gls{tRNA}
    \item \gls{miRNA}
    \item \gls{snRNA}
    \item \gls{scaRNA}
    \item \gls{snoRNA}
    \item \gls{scRNA}
    \item \gls{rRNA}
\end{itemize}

