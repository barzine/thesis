\chapter{Introduction}
\label{ch:Introduction}
\setlength{\epigraphwidth}{0.6\textwidth}
\setlength{\epigraphrule}{0.1pt}
%\epigraphhead[70]{%
    \epigraph{It would probably be oversimplifying the
matter,\\but I am strongly tempted to say,\\ ``All life is nucleic acid; the rest is
commentary''.}{\cite{asimov:WrongRel}}%
%}

\begin{comment}
When Asimov concluded with these words his chapter ``Beginning with bone'',
scientists had already started unfolding one of the most mesmerising biological
mystery: how does the Nature manage and organise the production of specific
effectors in specific locations. In other words how from one cell
\TK{revoir cette phrase: idea initiale pourquoi
organisme A different
de B, ou un foie et un foie et pas un coeur + comment d'une cellule-oeuf on a un
organisme complet qui se creée... ne pas s'étendre dessus par contre}.\
Watson and Crick by publishing the double-helix structure of \DNA\
\citep{DNA1953} unlocked our path to understand the natural ways of storing and
manipulating the information. As the whole past six decades were packed with major
discoveries and technological achievements, there are many, fascinating,
milestones that could be discussed.
The following pages present a summary of the facts and techniques
that form the biological and technical context of the work backing this thesis.

--- ici intro par rapport à Asimov et le fait que l'ADN est un 'blueprint' et qu'en
fonction du l'organe ou du develepment stage l'expression est régulé. On sait
maitenant que la régulation se passe à différent niveau.... finir peut être la
section sur Harvey qui a élucider la circulation sanguine car il a utilisé
l'observation et quantification du sang ---


As where the situation stands for now, we know that between the unique common
bluprint, which is the \DNA\ and the different phenotypes displayed by the tissues
comprising our bodies, they are many regulatory processes happening at different
layers (transcription, and translation).

 ### agent


\Rough{%
Gene expression, transcription, translation, regulation at each stage,
expected and reported correlation between transcripts and proteins
Measuring transcript expression by RNAseq, wet-lab part, analysis methods,
available large scale datasets on human tissues
Measuring protein expression by MS, data analysis – spectral counts, top 3, etc,
available large scale datasets on human tissues
The problem of comparing and integrating independent datasets, EBI’s GXA \\ \\
key concepts: DNA storage, RNA transfert of information and regulation,
protein effectors ==\textgreater phenotype.\\ \\
%
Do not forget the (tissues) specifications and the structural functions. =\textgreater Embryology.\\ \\
%
introduction real possible start: la soupe primaire -\textgreater creation de RNA, aa,
oeuf/poule
premiere cellule, \ldots (lire Darwin, il y a ptet moyen d ajouter une citation ou
quelque chose.%
}

\begin{itemize}
    \item Proteomics messy and noisy, challenging \ldots
    \item Transcriptomics seem a good proxy to study the phenotype => one kind of molecule made of the same pieces.
    \item Technology improved a lot from the microarrays to sequencing.
\end{itemize}

\end{comment}


\section{Diversity and universality of Life}




%\TK{Soupe primaire experience du Chimiste}
\TK{The central dogma of Molecular biology} => Crick describes the cycle

\fixme{Add references on replications, not going into that one}

\subsection{Translation + Regulation}

\subsection{Translation + Regulation}

Just so that the next part of the chapter got a OK presentation:
\begin{itemize}
    \item \gls{mRNA}
    \item \gls{ncRNA}
    \item \gls{tRNA}
    \item \gls{miRNA}
    \item \gls{snRNA}
    \item \gls{scaRNA}
    \item \gls{snoRNA}
    \item \gls{scRNA}
    \item \gls{rRNA}
\end{itemize}

\TK{Assumption mRNA and proteins : highly correlated [find references]}


\section{Capturing the expression in the laboratory}

Biological research uses mainly two approaches to
study the cell life intricacy and its underlying mechanisms.\\
The oldest approach is descriptive:


Small bits: corrections for \Rnaseq\ are already part of the analysis and
requires often as much flair than skills.
Contrarily to \Dnaseq\ where corrections can be applied \TK{add reference} and
then the analysis be done, in \Rnaseq each analysis requires a set of conform
quantification and normalisation methods.
While, there are quite established protocols for differential expression analysis,
there are presently many other downstream analyses that are cumbersome
and/or not settled yet. This is the case for this study.


Corrections for \Rnaseq\ are already part of the analysis and
requires often as much flair than skills.


\subsection{Transcriptomics Studies}
    \subsubsection{Main technologies}
While microarrays are measuring many \mRNAs\ at once, their number is limited,
    \subsubsection{A typical sequencing workflow}
\subsection{Proteomics}
    \subsubsection{Main technologies}
    \subsubsection{Mass sprectrometry}

\section{Big studies, big data and Reproducibility}


\subsection{Reproducibility issues}
        \begin{itemize}
            \item{Different samples}
            \item{Technology: wet lab but also software: Rupgrade, ....}
            \item{missing meta-data}
        \end{itemize}
    \subsection{Main concerns}
        \subsubsection{Detection}
        \subsubsection{Quantification}
    \subsection{Consistency through biological layers}

\begin{comment}

\subsection{Available big high-throughput studies in Human}

All the datasets with which I worked are fitting three main criteria.
They comprise human normal samples from at least three kind of tissues.
They have been sequenced with \Rnaseq\ for the transcriptomic studies or
analysed with Mass spectrometry for the proteomic ones.
The \emph{raw} data is available and reusable.

While they are quite a few more studies that I would have like to use on the
transcriptomic side, this last point was often the critical reason
why they have not been included.
Indeed, many times I encountered data with ambiguous encoding format and, as the
studies were a little bit outdated,
I also could not get the information from the original authors.

\subsubsection{Transcriptomics}

I describe hereafter the 5 transcriptomic datasets I used
in the chronological order of their first public release.
The \cref{tab:Trans5DF} summarises the main characteristics of the different
datasets.

\begin{sidewaystable}
        \centering
        \caption{\label{tab:Trans5DF}Technical description of the 5 transcriptomic
        dataset (\Rnaseq)
         used for this study}
    \begin{tabular}{@{}cccccccccc@{}}
    \toprule
    \multicolumn{1}{c|}
        {\multirow{2}{*}{ArrayExpress ID}} &
         \multicolumn{1}{c|}{\multirow{2}{*}{Data ID}} &
         \multicolumn{2}{c|}{\begin{tabular}[c]{@{}c@                                         {}}Library\\Preparation\end{tabular}} &
         \multicolumn{2}{c|}{Sequencing} &
         \multicolumn{2}{c|}{Replicates} &
         \multicolumn{1}{c|}{\multirow{2}{*}{\begin{tabular}[c]{@{}c@{}}Tissue\\
                 Number\end{tabular}}} &
         \multirow{2}{*}{\begin{tabular}[c]{@{}c@{}}Multi-sampling\\ from the \\ same         individual\end{tabular}} \\
         \cmidrule(lr){3-8}
         \multicolumn{1}{c|}{} & \multicolumn{1}{c|}{} &
         \multicolumn{1}{c|}{\begin{tabular}[c]{@{}c@{}}Whole\\ RNA\end{tabular}} &
         \multicolumn{1}{c|}{\begin{tabular}[c]{@{}c@{}}PolyA\\ selected\end{tabular}} &
         \multicolumn{1}{c|}{\begin{tabular}[c]{@{}c@{}}Single\\ end\end{tabular}} &
         \multicolumn{1}{c|}{\begin{tabular}[c]{@{}c@{}}Paired\\ end\end{tabular}} &
         \multicolumn{1}{c|}{Biological} & \multicolumn{1}{c|}{Technical} &
         \multicolumn{1}{c|}{} &  \\
    \midrule
    E-MTAB-305 & Castle & Y &  & Y &  &  &  & 11 &  \\
    E-GEOD-30352 & Brawand &  & Y & Y &  & Y &  & 8 &  \\
    E-MTAB-513 & IBM &  & Y & Y & Y &  & (Y) & 16 &  \\
    E-MTAB-2836 & Uhlén &  & Y &  & Y & Y & Y & 32 &  \\
    E-MTAB-2919 & Gtex  & Y &  &  & Y & Y &  & 54 & Y \\
    \bottomrule
    \end{tabular}
    \end{sidewaystable}



\paragraph{~Castle et al. dataset}

This dataset has been published along with the \paper{\citetitle{castleData}}
by \citet{castleData} who were interested to explore
with sequencing-based technology the whole RNA repertoire. They essentially
focused their study on the non coding part and found that
while \glspl{mRNA} could be highly tissue-specific, \glspl{ncRNA} have generally
greater tissue-specific expression patterns.

They used multiple-donors pooled tissues samples (purchased as total \gls{RNA})
and prepared the libraries following a whole transcriptomic protocol
\citep{Armour:2009}: where nonribosomal \gls{RNA} transcripts are
specifically amplified by \gls{PCR}.

They generated an average of 50 millions sequence reads per tissue
using an Illumina Genome Analyser-II sequencer (single-end).
They trimmed their original reads to 28 \gls{nt}
and released them through EMBL archives (\ENA{ERP000257}
and \ArrayExpress{E-MTAB-305}).

Despite several limitations (lack of replicates, old technology, small reads),
I used this dataset for two main reasons. It is the oldest available \Rnaseq\
data I found that was performed on Human normal tissues and it is comprising
\glspl{ncRNA}.

\paragraph{~Brawand et al. dataset}

In the corresponding article entitled \paper{\citetitle{VTpaper}},
\citet{VTpaper}~collected 6 organs from 10 different vertebrates:
9 mammalians (including Human) and a bird. They are focused on the
evolution of the mammalian transcriptomes -- while there were existing studies
on the matter, the sequencing approach was then creating new perspectives.

They have biological replicates: one male
and one female for the somatic tissue and two males for the testes samples.
They used a
polyA-selected protocol to prepare the libraries. Hence, the samples are largely
enriched in protein coding genes.

They generated an average of 3,2 billion reads of 76 base pairs per sample
using an Illumina Genome Analyser IIx (single-end) and they released them
through \gls{GEO} (accession number: GSE30352).
I personally retrieved the data from
\ArrayExpress{E-GEOD-30352}\footnote{ArrayExpress routinely imports
datasets from \gls{GEO} on a weekly basis.}.

\paragraph{~Illumina Body Map 2.0}

This dataset has been first created in 2010 and released in
2011\footnote{See: \citetitle{ibmEnsembl} - \cite{ibmEnsembl}} by Illumina
mostly to advertise its most recent technology improvement at that time:
the paired-end sequencing.
Until then, all the sequencing was done from only one end of the \gls{DNA} (or
\gls{cDNA}) fragments\footnote{Most of the following transcriptome
studies based on \Rnaseq\ are using paired-end sequencing.}.

The first published paper to analyse this dataset was done by
\citet{ibmrelatedpaper}: \paper{\citetitle{ibmrelatedpaper}}.
It was referenced many times since then as it was for a couple of years
the most extensive freely available \Rnaseq\ dataset of human tissues.

It comprises 16 tissues (one donor per tissue), which were prepared with a
polyA-selected library preparation protocol and then have been sequenced once
with a singled-end protocol and then a second time with a paired-end one. There
are some added libraries which have been created by mixing together the 16 tissues.
While each sample has been sequenced twice and that we have in principal
\emph{technical} replicates, these are not ``regular'' technical
replicates\footnote{\emph{Technical} replicates,
by contrast to \emph{biological} replicates,
usually imply that the same sample source and protocols have been used so the
error and the noise of a technique could be determined.}.

The sequencing was performed with an Illumina HiSeq 2000 and the reads were
released through \ArrayExpress{E-MTAB-503} (\ENA{ERP000546}).

\paragraph{~Uhlén et al. dataset}

The first version of this dataset (\ArrayExpress{E-MTAB-1733}) was published
as a part of \paper{\citetitle{Uhlen2014}} by \citet{Uhlen2014}. Then an extended
version (\ArrayExpress{E-MTAB-2836}), with new samples and 5 new tissues,
was released along with \paper{\citetitle{Uhlen2015}} \citep{Uhlen2015}.
Both papers are part of the
\href{http://www.proteinatlas.org/}{Human Protein Atlas}\footnote{%
\href{http://www.proteinatlas.org/}{http://www.proteinatlas.org/}}.
Uhlén et al.\ have created
an atlas revolving mostly around the spatial distribution of the proteins through
the Human body. They use many approaches and techniques which also include \Rnaseq.

They found that almost half of the proteins are expressed in all analysed tissues
(with an enrichment for the metabolism enzymes).

There are \emph{technical} and \emph{biological} replicates for the 32 tissues.
Except very few cases, the somatic tissues have male and female donors.

The libraries have been prepared following a polyA-selected protocols and have
been sequenced (paired-end) with an Illumina HiSeq 2000 or 2500.

I started to work with the first version, and when the extended version was
released, I included the new samples and tissues to my work.

\paragraph{~GTEx data}

The Genotype-Tissue Expression (\gls{GTEx}) project is funded by the NIH Common
Fund and aims to establish a resource database and associated tissue bank
for the study of the relationship between genetic variation and gene expression
and other molecular phenotypes in multiple reference tissues. The project was first
explained in a paper from the \cite{GTEx2013}: it consists to quickly collect
many tissues from postmortem donors so genotype-tissue expression analyses could
be done, notably \gls{eQTL} variants studies which study the modulation
of \gls{RNA} expression in function of \glspl{SNP}. The results of the
analyses are released through the GTEx portal (%
\href{http://gtexportal.org}{http://gtexportal.org}). The issue number 6235 of
the volume number 348 of \emph{Science} has published
many articles from this project. The most relevant to my work is
\paper{\citetitle{GTExTranscript}} from \cite{GTExTranscript}. While they study
the landscape of expression through the different tissues across the donors, they
put emphasis on the variation inter- and intra-individuals across the tissues.

As the project is quite ambitious and the collection and sequencing of the samples
are taking time, several freezes of the data have been released. My work is
including samples up to the fourth release of the pilot phase (v.1.4). This
release includes 54 tissue/cell type collected on 551 individuals.
The libraries were prepared from whole \gls{RNA} extracts and then sequenced
with a paired-end protocol on Illumina HiSeq 2000/2500 sequencers which produced
an average of 80 million reads.

The raw data is available for privacy reasons only through controlled access via
\dbGaP{phs000424.v4.p1} (access number specific to the version of the data I used
in my study). While getting access can take time, in principal every request for
academic research should be granted.

\subsubsection{Proteomics}

\paragraph{~Pandey data}

\cite{PandeyData} wanted to provide the first atlas for the human proteome.
Thus, they created the Human Proteome Map (%
\href{http://www.humanproteomemap.org/}{http://www.humanproteomemap.org/}) which
they released along their paper \paper{\citetitle{PandeyData}}.

To create their map, they have processed pooled samples from three individuals
per tissue type for 30 kind of histological normal human tissues and cell lines
(17 adult tissues, 7 fetal tissues and 6 haematopoietic cell types).

They used high-resolution Fourier-transform mass spectrometers
(LTQ-Orbitrap Elite and LTQ-Orbitrap Velos).

The different histological samples were fractionated first to protein level by
SDS–polyacrylamide gel electrophoresis (SDS–PAGE)
and then at the peptide level by basic reversed-phase liquid chromatography
(RPLC). They use state-of-art protocols to generate about 25 million
high-resolution tandem mass spectra from more
than 2,000 \gls{LC-MS/MS}.

To generate a high-quality data set, both precursorions and higher-energy
collisional dissociation (HCD)-derived fragment ions were measured
using the high-resolution and high-accuracy Orbitrapmass spectrometer.
Approximately 25 million high-resolution tandem mass spectra, acquired
from more than 2,000 LC-MS/MS (liquid chromatography followed by

\paragraph{~Kuster data}
\cite{KusterData}


\subsection{Co-studies on Transcriptomics and Proteomics in the literature}
\subsection{EBI Expression Atlas or how integrating independent datasets }


\section{Aims of the thesis}

One main focus of my work for this thesis was to appraise the
consistency of findings that have been enlightened in normal Human tissues by
individual large scale transcriptomics and, more recently, large-scale
proteomics studies.

While the amount of data to process and integrate can be challenging,
paradoxically the data available to draw statistically significant conclusions
is often still too sparse. However, we do have enough to provide at least a
qualitative assessment of the consistency and evaluate the feasibility of a
quantitative assessment in general, if not, realise a quantitative integration
for a few key tissues.









