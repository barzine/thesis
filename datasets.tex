\chapter{Available high-throughput normal human Datasets}
\label{ch:datasets}

In this chapter, I describe the data I used within my project.

All the datasets with which I worked are fitting three main criteria.
They comprise human normal samples from at least three kind of tissues.
They have been sequenced with \Rnaseq\ for the transcriptomic studies or
analysed with Mass spectrometry for the proteomic ones.
The \emph{raw} data is available and reusable.


\section{Transcriptomics}

While they are quite a few more studies that I would have like to use on the
transcriptomic side, this last point was often the critical reason
why they have not been included.
Indeed, many times I encountered data with ambiguous encoding format and, as the
studies were a little bit outdated,
I also could not get the information from the original authors.

I describe hereafter the 5 transcriptomic datasets I used
in the chronological order of their first public release.
The \cref{tab:Trans5DF} summarises the main characteristics of the different
datasets.

\begin{sidewaystable}
           \centering
           \caption{\label{tab:Trans5DF}Technical description of the 5 transcriptomic
           dataset (\Rnaseq)
            used for this study}
       \begin{tabular}{@{}cccccccccc@{}}
       \toprule
       \multicolumn{1}{c|}
           {\multirow{2}{*}{ArrayExpress ID}} &
            \multicolumn{1}{c|}{\multirow{2}{*}{Data ID}} &
            \multicolumn{2}{c|}{\begin{tabular}[c]{@{}c@                                               {}}Library\\Preparation\end{tabular}} &
            \multicolumn{2}{c|}{Sequencing} &
            \multicolumn{2}{c|}{Replicates} &
            \multicolumn{1}{c|}{\multirow{2}{*}{\begin{tabular}[c]{@{}c@{}}Tissue\\
                    Number\end{tabular}}} &
            \multirow{2}{*}{\begin{tabular}[c]{@{}c@{}}Multi-sampling\\ from the \\ same               individual\end{tabular}} \\
            \cmidrule(lr){3-8}
            \multicolumn{1}{c|}{} & \multicolumn{1}{c|}{} &
            \multicolumn{1}{c|}{\begin{tabular}[c]{@{}c@{}}Whole\\ RNA\end{tabular}} &
            \multicolumn{1}{c|}{\begin{tabular}[c]{@{}c@{}}PolyA\\ selected\end{tabular}} &
            \multicolumn{1}{c|}{\begin{tabular}[c]{@{}c@{}}Single\\ end\end{tabular}} &
            \multicolumn{1}{c|}{\begin{tabular}[c]{@{}c@{}}Paired\\ end\end{tabular}} &
            \multicolumn{1}{c|}{Biological} & \multicolumn{1}{c|}{Technical} &
            \multicolumn{1}{c|}{} &  \\
       \midrule
       E-MTAB-305 & Castle & Y &  & Y &  &  &  & 11 &  \\
       E-GEOD-30352 & Brawand &  & Y & Y &  & Y &  & 8 &  \\
       E-MTAB-513 & IBM &  & Y & Y & Y &  & (Y) & 16 &  \\
       E-MTAB-2836 & Uhlén &  & Y &  & Y & Y & Y & 32 &  \\
       E-MTAB-2919 & Gtex  & Y &  &  & Y & Y &  & 54 & Y \\
       \bottomrule
       \end{tabular}
       \end{sidewaystable}

\subsection{Castle et al. dataset}

This dataset has been published along with the \paper{\citetitle{castleData}}
by \citet{castleData} who were interested to explore
with sequencing-based technology the whole RNA repertoire. They essentially
focused their study on the non coding part and found that
while \glspl{mRNA} could be highly tissue-specific, \glspl{ncRNA} have generally
greater tissue-specific expression patterns.

They used multiple-donors pooled tissues samples (purchased as total \gls{RNA})
and prepared the libraries following a whole transcriptomic protocol
\citep{Armour:2009}: where nonribosomal \gls{RNA} transcripts are
specifically amplified by \gls{PCR}.

They generated an average of 50 millions sequence reads per tissue
using an Illumina Genome Analyser-II sequencer (single-end).
They trimmed their original reads to 28 \gls{nt}
and released them through EMBL archives (\ENA{ERP000257}
and \ArrayExpress{E-MTAB-305}).

Despite several limitations (lack of replicates, old technology, small reads),
I used this dataset for two main reasons. It is the oldest available \Rnaseq\
data I found that was performed on Human normal tissues and it is comprising
\glspl{ncRNA}.

\subsection{Brawand et al. dataset}

In the corresponding article entitled \paper{\citetitle{VTpaper}},
\citet{VTpaper}~collected 6 organs from 10 different vertebrates:
9 mammalians (including Human) and a bird. They are focused on the
evolution of the mammalian transcriptomes -- while there were existing studies
on the matter, the sequencing approach was then creating new perspectives.

They have biological replicates: one male
and one female for the somatic tissue and two males for the testes samples.
They used a
polyA-selected protocol to prepare the libraries. Hence, the samples are largely
enriched in protein coding genes.

They generated an average of 3,2 billion reads of 76 base pairs per sample
using an Illumina Genome Analyser IIx (single-end) and they released them
through \gls{GEO} (accession number: GSE30352).
I personally retrieved the data from
\ArrayExpress{E-GEOD-30352}\footnote{ArrayExpress routinely imports
datasets from \gls{GEO} on a weekly basis.}.

\subsection{Illumina Body Map 2.0}
This dataset has been first created in 2010 and released in
2011\footnote{See: \citetitle{ibmEnsembl} - \cite{ibmEnsembl}} by Illumina
mostly to advertise its most recent technology improvement at that time:
the paired-end sequencing.
Until then, all the sequencing was done from only one end of the \gls{DNA} (or
\gls{cDNA}) fragments\footnote{Most of the following transcriptome
studies based on \Rnaseq\ are using paired-end sequencing.}.

The first published paper to analyse this dataset was done by
\citet{ibmrelatedpaper}: \paper{\citetitle{ibmrelatedpaper}}.
It was referenced many times since then as it was for a couple of years
the most extensive freely available \Rnaseq\ dataset of human tissues.

It comprises 16 tissues (one donor per tissue), which were prepared with a
polyA-selected library preparation protocol and then have been sequenced once
with a singled-end protocol and then a second time with a paired-end one. There
are some added libraries which have been created by mixing together the 16 tissues.
While each sample has been sequenced twice and that we have in principal
\emph{technical} replicates, these are not ``regular'' technical
replicates\footnote{\emph{Technical} replicates,
by contrast to \emph{biological} replicates,
usually imply that the same sample source and protocols have been used so the
error and the noise of a technique could be determined.}.

The sequencing was performed with an Illumina HiSeq 2000 and the reads were
released through \ArrayExpress{E-MTAB-503} (\ENA{ERP000546}).


\subsection{Uhlén et al. dataset}

The first version of this dataset (\ArrayExpress{E-MTAB-1733}) was published
as a part of \paper{\citetitle{Uhlen2014}} by \citet{Uhlen2014}. Then an extended
version (\ArrayExpress{E-MTAB-2836}), with new samples and 5 new tissues,
was released along with \paper{\citetitle{Uhlen2015}} \citep{Uhlen2015}.
Both papers are part of the
\href{http://www.proteinatlas.org/}{Human Protein Atlas}\footnote{%
\href{http://www.proteinatlas.org/}{http://www.proteinatlas.org/}}.
Uhlén et al.\ have created
an atlas revolving mostly around the spatial distribution of the proteins through
the Human body. They use many approaches and techniques which also include \Rnaseq.

They found that almost half of the proteins are expressed in all analysed tissues
(with an enrichment for the metabolism enzymes).

There are \emph{technical} and \emph{biological} replicates for the 32 tissues.
Except very few cases, the somatic tissues have male and female donors.

The libraries have been prepared following a polyA-selected protocols and have
been sequenced (paired-end) with an Illumina HiSeq 2000 or 2500.

I started to work with the first version, and when the extended version was
released, I included the new samples and tissues to my work.

\subsection{GTEx data}

The Genotype-Tissue Expression (\gls{GTEx}) project is funded by the NIH Common
Fund and aims to establish a resource database and associated tissue bank
for the study of the relationship between genetic variation and gene expression
and other molecular phenotypes in multiple reference tissues. The project was first
explained in a paper from the \cite{GTEx2013}: it consists to quickly collect
many tissues from postmortem donors so genotype-tissue expression analyses could
be done, notably \gls{eQTL} variants studies which study the modulation
of \gls{RNA} expression in function of \glspl{SNP}. The results of the
analyses are released through the GTEx portal (%
\href{http://gtexportal.org}{http://gtexportal.org}). The issue 6235 of
\emph{Science} comprises
many articles from this project. The most relevant to my work is
\paper{\citetitle{GTExTranscript}} from \cite{GTExTranscript}. While they study
the landscape of expression through the different tissues across the donors, they
put emphasis on the variation inter- and intra-individuals across the tissues.

As the project is quite ambitious and the collection and sequencing of the samples
are taking time, several freezes of the data have been released. My work is
including samples up to the fourth release of the pilot phase (v.1.4). This
release includes 54 tissue/cell type collected on 551 individuals.
The libraries were prepared from whole \gls{RNA} extracts and then sequenced
with a paired-end protocol on Illumina HiSeq 2000/2500 sequencers which produced
an average of 80 million reads.

The raw data is available for privacy reasons only through controlled access via
\dbGaP{phs000424.v4.p1} (access number specific to the version of the data I used
in my study). While getting access can take time, in principal every request for
academic research should be granted.

\section{Proteomics}

In the issue 7502 of \textit{Nature} (2014), two teams of authors released their own
\emph{``draft of the human proteome''} based on the study of several tissues
with \ms. There were already other human protein maps available,
e.g.\ the Human protein atlas (%
\href{http://www.humanproteomemap.org/}{www.proteinatlas.org/}), but those one
were mostly reporting the spatial expression of the proteins (as they were based
on immunohistochemistry or other means of identification) than
quantifying their abundance in each tissue.


\subsection{Pandey data}

\cite{PandeyData} created the Human Proteome Map (%
\href{http://www.humanproteomemap.org/}{\small www.humanproteomemap.org/}) which
they released along their paper \paper{\citetitle{PandeyData}}.

To create their map, they have processed 30 kind of histological normal human
tissues and cell line (17 adult tissues, 7 fetal tissues and 6 haematopoietic
cell types) samples. The samples were created from pooled samples of three
individuals (generally two males and one female).

They used a label-free method of library preparation to quantify as many proteins
they could.

They first fractionated the samples to protein level by
\gls{SDS-PAGE} and then at peptide level by \gls{RPLC}. Then, they use
state-of-art \gls{MS/MS} protocols (with high-resolution and high accuracy
\glspl{FTMS}: a LTQ-Orbitrap Elite and then a LTQ-Orbitrap Velos).
They generated about 25 million
high-resolution mass spectra which account for over 2,000 \gls{LC-MS/MS} profiles.

While their work to generate the data was highly appraise, the identification and
quantification data they provided the community was criticised
\citep{Ezkurdia2014-qx}.


\subsection{Kuster data}
\cite{KusterData}




\subsection{Cutler data}
