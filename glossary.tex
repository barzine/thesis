%% acronyms
\newacronym{RNA}{RNA}{\textbf{R}ibo\textbf{n}ucleic \textbf{A}cid}
\newacronym{DNA}{DNA}{\textbf{D}eoxyribo\textbf{n}ucleic \textbf{A}cid}
\newacronym{PCR}{PCR}{\textbf{P}olymerase \textbf{C}hain \textbf{R}eaction}
\newacronym{EBI}{EBI}{\textbf{E}uropean \textbf{B}ioinformatics \textbf{I}nstitute}
%\newacronym{mRNA}{\textit{m}RNA}{\textit{\textbf{m}essenger}
%\textbf{R}ib\textbf{n}ucleic \textbf{a}cid}
\newacronym{GEO}{GEO}{\textbf{G}ene \textbf{E}xpression \textbf{O}mnibus}
\newacronym{HCD}{HCD}{\textbf{H}igher-energy \textbf{C}ollisional
\textbf{D}issociation}

\newacronym{MAD}{MAD}{\textbf{M}edian \textbf{A}bsolute \textbf{D}eviation}

\newacronym{HLA}{HLA}{\textbf{H}uman \textbf{l}eukocyte \textbf{a}ntigen}
\newacronym{SILAC}{SILAC}{Stable Isotope Labeling by Amino acids in Cell culture}
\newacronym{iTRAQ}{iTRAQ}{\textbf{i}sobaric \textbf{t}ag
for \textbf{r}elative and \textbf{a}bsolute \textbf{q}uantification}
\newacronym{ICAT}{ICAT}{Isotope-Coded Affinity Tag}
\newacronym{TMT}{TMT™}{Tandem Mass Tags}
\newacronym{2D-DIGE}{2D-DIGE}{2D-Differential In-Gel Electrophoresis}
\newacronym{IBAQ}{IBAQ}{{I}ntensity {B}ased {A}bsolute {Q}uantification}
\newacronym{XIC}{XIC}{Extracted-ion current}
\newacronym{RIC}{RIC}{Reconstructed-ion chromatagram}
\newacronym{ppm}{ppm}{part per million}
\newacronym{ID}{ID}{Identification number}
\newacronym{SRM}{SRM}{Selected Reaction Monitoring}
\newacronym{PRM}{PRM}{Parallel Reaction Monitoring}
\newacronym{MRM}{MRM}{Multiple Reaction Monitoring}
\newacronym{AUC}{AUC}{Area Under the Curve}
\newacronym{CID}{CID}{\textbf{C}ollision-\textbf{I}nduced \textbf{D}issociation}
\newacronym{ETD}{ETD}{\textbf{E}lectron \textbf{T}ransfer \textbf{D}issociation}

\newabbreviation[description={\textit{m}essenger \glsxtrshort{RNA}}]
{mRNA}{mRNA}{messenger RNA}
\newabbreviation[description={Dalton} is a unified atomic mass unit. It may be also annotated as \textbf{u}]{Da}{Da}{Dalton}

\newacronym{laser}{laser}{light amplification by stimulated emission of radiation}

\newacronym{SVM}{SVM}{Support Vector Machine}

\newacronym{RPKM}{RPKM}{\textbf{R}eads \textbf{P}er \textbf{K}ilobase of a
feature (\ie\ transcript in most cases) per \textbf{M}illion mapped
\textbf{R}eads}

\newacronym{FPKM}{FPKM}{\textbf{F}ragments \textbf{P}er \textbf{K}ilobase of
a feature (\ie\ transcript in most cases) per \textbf{M}illion mapped
\textbf{F}ragments}

\newacronym{TMM}{TMM}{\textbf{T}rimmed \textbf{m}ean of \textbf{M} values (M: log expression ratios)}

\newacronym{PPKM}{PPKM}{\glsxtrshort{PSM}s \textbf{P}er \textbf{K}ilobase of gene per \textbf{M}illion}

\newacronym{FDR}{FDR}{{F}alse {d}iscovery {r}ate}

\newacronym{SNR}{SNR}{{S}ignal-to-{N}oise {R}atio}
\newacronym{PSM}{PSM}{{P}eptide {S}pectrum {M}atch}
\newacronym{PEP}{PEP}{{P}osterior {E}rror {P}robability}


\newacronym{PTM}{PTM}{\textbf{P}ost-\textbf{T}ranslational \textbf{M}odification}

\newacronym{SAM}{SAM}{\textbf{S}equence \textbf{Alignment}/\textbf{M}ap}

\newacronym{EM}{EM}{\textbf{E}xpectation-\textbf{M}aximisation}

\newacronym{FASP}{FASP}{Filter-Aided Sample Preparation}

\newacronym{UPLC}{UPLC}{ultra performance liquid chromatography}

\newacronym{HPLC}{HPLC}{high-performance liquid chromatography}

\newacronym{pH}{pH}{potentiel Hydrogen}

\newacronym{TOF}{TOF}{time-of-flight}
\newacronym{LTQ}{LTQ}{linear trap quadrupole}
\newacronym{ICR}{ICR}{ion cyclotron}
\newacronym{LIT}{LIT}{linear ion trap}

\newacronym{TB}{TB}{\textbf{T}era\textbf{B}yte}

\newacronym{IBM}{IBM}{Illumina Body Map 2.0}

\newacronym{TREP}{TREP}{Tissue Reference Expression Profile}

\newabbreviation[longplural={gene set enrichment analyses}]{gsea}{GSEA}{Gene set enrichment analysis}
\newabbreviation[longplural={\glsxtrshort{go} enrichment analyses}]{goa}{GOA}{\glsxtrshort{go} enrichment analysis}
\newabbreviation{go}{GO}{Gene Ontology}
\newabbreviation[longplural={differential expression analyses}]{DEA}{DEA}{Differential expression analysis}
\newabbreviation[longplural={differential gene expression analyses}]{DGEA}{DGEA}{Differential Gene Expression Analysis}
\newabbreviation[longplural={over-representation analyses}]{ORA}{ORA}{over-representation analysis}
\newabbreviation[description={Peptide Mass Fingerprinting}]{PMF}{PMF}{peptide mass fingerprinting}
\newabbreviation[description={Peptide Fragment Fingerprinting}]{PFF}{PFF}{peptide fragment fingerprinting}
\newabbreviation%
[description={\textit{B}GZF-compressed
\glsxtrshort{SAM}}]{BAM}{BAM}{BGZF-compressed SAM}

\newabbreviation%
[description={\textit{R}adio \textit{f}requency}]{RF}{RF}{radio frequency}

\newabbreviation%
[description={Olfactory receptor}]{OR}{OR}{olfactory receptor}

\newabbreviation%
[description={\textit{D}irect \textit{c}urrent}]{DC}{DC}{direct current}
%\newacronym{ncRNA}{\textit{nc}RNA}{\textbf{n}on \textbf{c}oding \textbf{R}ibo\textbf{n}ucleic \textbf{A}cid}

\newacronym{NIH}{NIH}{\textbf{N}ational \textbf{I}nstitutes of \textbf{H}ealth
(USA)}

\newacronym{FFPE}{FFPE}{\textbf{F}ormalin-\textbf{F}ixed
\textbf{P}araffin-\textbf{E}mbedded}

\newacronym{FF}{FF}{\textbf{F}resh-\textbf{F}rozen}

\newacronym{CAGE}{CAGE}{Cap analysis gene expression}
\newabbreviation%
[description={\textit{N}on-\textit{c}oding \glsxtrshort{RNA}}]
{ncRNA}{ncRNA}{Non-coding RNA}

\newabbreviation%
[description={\textit{L}ong \textit{N}on \textit{c}oding \glsxtrshort{RNA}}]
{lncRNA}{lncRNA}{Long non-coding RNA}

\newabbreviation%
[description={\textit{S}mall \textit{n}ucle\textbf{o}lar \glsxtrshort{RNA}}]
{snoRNA}{snoRNA}{Small nucleolar RNA}

\newabbreviation%
[description={\textit{S}mall \textit{Ca}jal body \glsxtrshort{RNA}}]
{scaRNA}{scaRNA}{Small Cajal body RNA}

\newabbreviation%
[description={\textit{S}mall \textit{n}uclear ribonucleic \glsxtrshort{RNA}}]
{snRNA}{snRNA}{Small nuclear ribonucleic RNA}

\newabbreviation%
[description={\textit{S}mall \textit{c}onditional \glsxtrshort{RNA}}]
{scRNA}{scRNA}{Small conditional RNA}

\newabbreviation%
[description={\textit{T}ransfer \glsxtrshort{RNA}}]
{tRNA}{tRNA}{Transfer RNA}

\newabbreviation%
[description={\textit{c}omplementary \glsxtrshort{DNA}}]
{cDNA}{cDNA}{complementary DNA}

\newabbreviation%
[description={micro\glsxtrshort{RNA}}]
{miRNA}{miRNA}{microRNA}

\newabbreviation%
[description={Ribosomal \glsxtrshort{RNA}}]
{rRNA}{rRNA}{Ribosomal RNA}

\newabbreviation%
[description={\textit{d}eoxynucleoside \textit{t}ri\textit{p}hosphate.
The N indicates that it can be any
nucleotide (usually either Adenosine (A), Cytosine (C), Guanine (G) or
Thymine (T))}]{dNTP}{dNTP}{deoxynucleoside triphosphate}

\newabbreviation[description={\textit{C}oefficient of \textit{v}ariation}]%
{cv}{cv}{coefficient of variation}

\newabbreviation[description={\textit{S}tandard \textit{d}eviation}]%
{sd}{sd}{standard deviation}

\newabbreviation[description={\textit{Var}iation}]%
{var}{var}{variation}

\newabbreviation%
[description={\textit{T}issue-\textit{S}pecific}]
{TS}{TS}{Tissue-Specific}
%%%% glossary entries
\newglossaryentry{RNA-Seq}{name={RNA-Seq},
    description={RNA sequencing, which can also be called whole transcriptome
    shotgun sequencing}}

\newglossaryentry{DNA-Seq}{name={DNA-Seq},
    description={DNA sequencing, which can also be called whole genome
    shotgun sequencing}}


\newglossaryentry{CAGE-Seq}{name={CAGE-Seq},
description={\glsxtrshort{CAGE} Sequencing}}

\newglossaryentry{miRNA-Seq}{name={miRNA-Seq},
    description={miRNA sequencing, which can also be called miRNA shotgun
    sequencing}}

\newglossaryentry{HBB}{name={HBB},description={Hemoglobin Subunit Beta}}
\newglossaryentry{qPCR}{name={qPCR}, description={quantitative real-time
\textbf{P}olymerase \textbf{C}hain \textbf{R}eaction}}

\newglossaryentry{RT-qPCR}{name={RT-qPCR},
description={\textbf{R}everse-\textbf{T}ranscription
\textbf{q}uantitative real-time \textbf{P}olymerase \textbf{C}hain
\textbf{R}eaction is a molecular biology technique to quantify the amount of
\glsxtrlong{RNA} in a given cell or sample. Cycles of monitored replications are
used to robustly measure the gene expression. It is often considered
to be the most powerful and sensitive of the quantitative assay for
\glsxtrlong{RNA}. However, this method requires to know in advance which are the
genes of interest.}}

\newglossaryentry{ArrayExpress}{name=ArrayExpress,description={EBI archive of
Functional Genomics Data. It stores data from high-throughput functional
genomics experiments, and provides these data for reuse to the
research community.}}

\newglossaryentry{ENA}{name=ENA,description={The European Nucleotide Archive
provides a comprehensive record of the world's nucleotide sequencing information,
covering raw sequencing data, sequence assembly information and functional
annotation.}}

\newglossaryentry{Ensembl}{name=Ensembl,description={Database that is the joint
project between EMBL-EBI and the Wellcome Trust Sanger Institute to develop a
software system which produces and maintains automatic annotation on selected
eukaryotic genomes.}}

\newglossaryentry{dbGaP}{name=dbGaP,description={The database of Genotypes and
\Glspl{phenotype} (dbGaP) was developed to archive and distribute the data and results
from studies that have investigated
the interaction of genotype and \gls{phenotype} in Humans.}}

\newglossaryentry{Pride}{name=PRIDE,description={PRoteomics IDEntifications
(PRIDE) database is a centralized, standards compliant, public data repository
for proteomics data, including protein and peptide identifications,
post-translational modifications and supporting spectral evidence.
PRIDE is a core member in the ProteomeXchange (PX) consortium,
which provides a single point for submitting mass spectrometry based proteomics
data to public-domain repositories.
Datasets are submitted to PRIDE via ProteomeXchange and are handled by expert
biocurators.}}

\newglossaryentry{Proteomicsdb}{name=Proteomicsdb,description={ProteomicsDB is a
joint effort of the Technische Universität München (TUM) and SAP SE\@. It is
dedicated to expedite the identification of the human proteome and its use across
the scientific community.}}

\newglossaryentry{Uniprot}{name=UniProt,description={The \textbf{Uni}versal
\textbf{Prot}ein Resource provides the scientific community with a comprehensive
high-quality and freely accessible resource for protein sequence and functional
annotation data.}}

\newglossaryentry{GTEx}{name=GTEx, description={The \textbf{G}enotype-\textbf{T}issue
\textbf{Ex}pression project establishes a resource database and associated tissue
bank for the scientific community to study the relationship between genetic
variation and gene expression in human tissues.}}

\newabbreviation{TCGA}{TCGA}{The Cancer Genome Atlas}

\newabbreviation{pcawg}{PCAWG}{Pan-Cancer Analysis of Whole Genomes}

\newglossaryentry{TIGER}{name=TiGER,
description={\textbf{Ti}ssue-specific \textbf{G}ene \textbf{E}xpression and \textbf{R}egulation is a database created by the Bioinformatics lab at Wilmer
Institute, Johns Hopkins University}}

\newglossaryentry{nt}{name={nt},description={\textbf{N}ucleo\textbf{t}id;
common unit of length for single-stranded nucleic acids}}

\newglossaryentry{bp}{name={bp},description={\textbf{B}ase {P}air;
unit of length for double-stranded nucleic acids}}

\newglossaryentry{SNP}{name={SNP},description={Single Nucleotide Polymorphism}}

\newglossaryentry{eQTL}{name={eQTL},
  description={expression Quantitative Trait Locus},
  %first={\glsentrydesc{eQTL} (\glsentrytext{eQTL})},
  plural={eQTL},
  descriptionplural={expression Quantitative Trait Loci},
  %firstplural={\glsentrydescplural{LED} (\glsentryplural{})}
}

\newacronym{EST}{EST}{Expressed sequence tag}

\newglossaryentry{GTF}{name={GTF},
    description={\textbf{G}ene \textbf{T}ransfer \textbf{F}ormat is a tab-delimeted
    file format based on the \glsxtrshort{GFF} format and hold information about
    gene structure. A main feature of this file is that it can be validatable
    which increases the data reliability.}}

\newglossaryentry{GFF}{name={GFF},
    description={\textbf{G}eneral \textbf{F}eature \textbf{F}ormat is a tab-delimited
    file format that records gene information or other features as \glsxtrshort{DNA},
\glsxtrshort{RNA} or protein sequences.}}


\newglossaryentry{SeqcMa}{name={SEQC/MAQC-III},
    description={The third phase of the \glsxtrshort{MAQC} project (MAQC-III),
    also called Sequencing Quality Control (SEQC),
    aimed at assessing the technical performance of next-generation sequencing
    platforms by generating benchmark datasets with reference samples and
    evaluating advantages and limitations of various bioinformatics strategies
    in \glsxtrshort{RNA} and \glsxtrshort{DNA} analyses.}}


\newacronym{SEQC}{SEQC}{\textbf{SE}quencing \textbf{C}ontrol \textbf{Q}uality}
\newacronym{MAQC}{MAQC}{\textbf{M}icro\textbf{A}rray
\textbf{Q}uality \textbf{C}ontrol}


\newabbreviation{LC}{LC}{Liquid Chromatography}
\newabbreviation{MS}{MS}{Mass Spectrometry}
\newabbreviation[description={\glsxtrlong{LC} (LC)
followed by tandem \glsxtrshort{MS}}]%
{LC-MS/MS}{LC-MS/MS}{Liquid Chromatography (LC) followed by tandem Mass
Spectrometry}

\newabbreviation[description={tandem \glsxtrshort{MS}}]
{MS/MS}{MS/MS}{tandem MS}

\newabbreviation[description={\glsxtrlong{LC} (LC)
followed by \glsxtrshort{MS}}]%
{LC-MS}{LC-MS}{Liquid Chromatography (LC) followed by Mass Spectrometry}

\newglossaryentry{Bottom-up}{name={Bottom-up approach},
text={bottom-up},
description={In proteomics, bottom-up approaches (in contrast to
\emph{top-down} approaches) involve a step of proteolytic digestion prior
to the Mass Spectrometry analysis.}}

\newglossaryentry{Top-down}{name={Top-down approach},
text={top-down},
description={In proteomics, top-down approaches (in contrast to the
\emph{bottom-up} approaches) analyses directly the proteins without any prior
step of proteolytic digestion before the Mass Spectrometry analysis (which often
involves \glsxtrshort{MS/MS})}}

\newacronym{FTMS}{FTMS}{\textbf{F}ourier \textbf{t}ransform \textbf{M}ass
\textbf{s}pectrometer}

\newabbreviation[description={\textit{F}ourier \textit{t}ransfo}]
{FT}{FT}{Fourier transform}

\newabbreviation[description={\textit{D}ata-\textit{d}ependant
\textit{a}cquisition}]{DDA}{DDA}{data-dependent acquisition}

\newabbreviation[description={\textit{D}ata-\textit{i}ndependant
\textit{a}cquisition}]{DIA}{DIA}{data-independent acquisition}

\newabbreviation[description={\textit{C}odant \glsxtrshort{DNA} \textit{s}equence}]
{CDS}{CDS}{coding DNA sequence}

\newacronym{ESI}{ESI}{Electron Spray Ionisation}
\newacronym{MALDI}{MALDI}{matrix-assisted \glsxtrshort{laser} desorption ionisation}

\newacronym{SDS}{SDS}{\textbf{S}odium \textbf{D}odecyl \textbf{S}ulfate}
\newacronym{PAGE}{PAGE}{\textbf{P}oly\textbf{a}crylamide \textbf{G}el
\textbf{E}lectrophoresis}
\newabbreviation[description={\glsxtrlong{SDS}-\glsxtrlong{PAGE}}]
{SDS-PAGE}{SDS-PAGE}{Sodium Dodecyl Sulphate-\gls{PAGE}}


\newacronym{LDS}{LDS}{\textbf{L}ithium \textbf{D}odecyl \textbf{S}ulfate}
\newabbreviation[description={\glsxtrlong{LDS}-\glsxtrlong{PAGE}}]
{LDS-PAGE}{LDS-PAGE}{Lithium Dodecyl Sulphate-\gls{PAGE}}


\newabbreviation[description={\textbf{R}eversed-\textbf{P}hase
\glsxtrshort{LC}}]
{RPLC}{RPLC}{Reversed-phase LC}


\newglossaryentry{Refseq}{name=RefSeq,description={The \textbf{Ref}erence
\textbf{Seq}uence database is an open access, annotated and curated
collection of publicly available nucleotide sequences (\glsxtrshort{DNA},
\glsxtrshort{RNA}) and their protein products.}}


\newglossaryentry{Phred}{name=Phred,description={A Phred quality score is a
measure of the quality of the identification of the nucleobases generated by
automated DNA sequencing.}}


\newglossaryentry{GRCh}{name=GRCh, description={\textbf{G}enome \textbf{R}eference
\textbf{C}onsortium \textbf{h}uman genome build. It is always followed by a
version number.}}

\newglossaryentry{Perl}{name={Perl}, description={Perl langage is
an open-source \textit{interpreted} programming langage developed by Larry Wall
and first released in 1987 to easily handle textual information.}}

\newglossaryentry{Python}{name={Python}, description={Python langage is a
\textit{high-level} and \textit{interpreted} programming langage developed by
Guido van Rossum and first released in 1991. The main purpose of this langage is
to be multivalent while enhancing code readibility.}}

\newglossaryentry{R}{name={R}, description={R langage is an open-source
\textit{interpreted} programming langage Ross Ihaka and Robert Gentleman and the
first oficial release was in 1995. \textit{R} provides support for statistical
computing.}}

\newglossaryentry{BiocR}{name={Bioconductor},description={Bioconductor provides
tools for the analysis and comprehension of high-throughput genomic data.
Bioconductor uses the R statistical programming language, and is open source
and open development}}

\newglossaryentry{cluster}{name={Computer cluster}, text={computer cluster},
description={Set of connected
computers that work together to improve performance over a single computer}}

\newglossaryentry{GENCODE}{name={GENCODE},description={Project that produces
high quality reference gene annotation for human and mouse genomes.}}

\newglossaryentry{ENCODE}{name={ENCODE},description={The ENCODE (Encyclopedia
of DNA Elements) Consortium is an international collaboration of research
groups funded by the \glsxtrlong{NHGRI}}}
\newacronym{NHGRI}{NHGRI}{National Human Genome Genome Research Institute}

\newglossaryentry{FASTQ}{name={FASTQ},description={Text-based file format.
It records for
each cluster read, a unique identifier, a nucleotide sequence and the call
accuracy for each base (\gls{Phred} score). Optionally, there can be more
information, \eg\ the spatial position of the cluster on the flow cell.}}


\newglossaryentry{Fasta}{name={FASTA},description={Text-based file format
standard. The file records protein or nucleotide sequences using their respective
single letter codes.}}


\newglossaryentry{flow}{name={Flow cell},text={flow cell},
description={Support of Illumina
sequencing. It enables the parallelisation of the sequencing of millions of
\gls{DNA} fragments together which are kept spatially separated in clusters.
It is a glass slide with lanes and each lane is coated with short nucleotide
sequences that are used to hybridise by complementarity adapters on the \gls{DNA}
that will be sequenced.}}


\newglossaryentry{fusionG}{name={Fusion gene}, text={fusion gene},
description={It is a gene that is
the product of the fusion of parts of two different genes.}}

\newglossaryentry{Pkcap}{name={Peak capacity}, text={peak capacity},
description={In Liquid Chromatography, the \emph{peak capacity} defines the
efficiency of a column to isolate componants from an initial mixture.}}

\newglossaryentry{ampholyte}{name={Ampholyte}, text={ampholyte},
description={Molecule which can both gives or accepts a proton (H$^+$).}}


%\newglossaryentry{p-value}{name={p-value},text={p-value},description={TBA}}

%\newglossaryentry{q-value}{name={q-value},text={q-value},description={TBA}}

\newglossaryentry{FANTOM5}{name=FANTOM5,
description={FANTOM5 is a consortium that systematically investigated
the genes expressed in all cell types the human body and the genomic regions that
contains the \glsxtrlong{TSS}.}}

\newacronym{TSS}{TSS}{Transcription starting site}

\newacronym{HTS}{HTS}{Hight-Throughput Sequencing}

\newabbreviation[description={This atlas is a Swedish-based program
which aims to map all the human proteins in cells,
tissues and organs using integration of various omics technologies}]%
{HPA}{HPA}{Human Protein Atlas}

\newglossaryentry{phenotype}{name={phenotype},
description={Set of observable characteristics or traits
of an individual. The traits can be inherited (genotype),
due to the environment or from the interaction of the environment with the genotype.}}

\newacronym{aa}{aa}{amino acid}


\newabbreviation[description={{T}arget {D}ecoy search {A}pproach}]%
{TDA}{TDA}{target-decoy search approach}

\newacronym{ROC}{ROC}{{R}eceveir {O}perating {C}haracteristic}

\newglossaryentry{NSAF}{name={NSAF},
description={normalized spectral abundance factor}}

\newacronym{NP}{NP}{nondeterministic polynomial time}

\newabbreviation[description={protein-to-\glsxtrshort{mRNA} ratio}]{PTR}{PTR}{protein-to-mRNA}
\newabbreviation[description={relative protein-to-\glsxtrshort{mRNA} ratio}]{rPTR}{rPTR}{relative protein-to-mRNA}
