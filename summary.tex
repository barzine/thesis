\clearpage
\chapter*{Summary}\label{ch:summary}
\addcontentsline{toc}{chapter}{Summary}
\vspace{-1cm}
\begin{singlespace}
    %\textbf{Investigating Normal Human Gene Expression in Tissues
    %with High-throughput Transcriptomic and Proteomic Data.}
    {\small With the improvement of high-throughput technologies
    during the last decade,
    several studies exploring the normal gene expression in human tissues
    have been published.
    Many studies examine the transcriptome with RNA sequencing (RNA-Seq),
    and others probe the proteome with unlabelled bottom-up Mass Spectrometry.
    As the sampling of undiseased tissues is difficult,
    the community often refers to expression atlases,
    which are collating these studies,
    to support or validate new findings.
    Despite many overlapping tissues between the studies,
    few atlases attempt to integrate all the data together in any way.\mybr\

    In this thesis, I investigate the consistency of gene expression
    across tissues and studies.
    Initially, my study's aim was
    the integration and comparison of transcriptomic data
    (generated with RNA-Seq) only.
    Then, with the release of two large Mass Spectrometry (MS) proteomics studies,
    the scope of my integration broadened to
    the comparison of the transcriptomic and proteomic data.\mybr\

    After describing the transcriptomic and proteomic data
    and their state-of-art processing (\Cref{ch:datasets}),
    I review several identified sources of biases
    and my approaches to limit their effects (\Cref{ch:expression}).\mybr\

    The integration of the various transcriptomic datasets (\Cref{ch:Transcriptomics})
    shows that
    the biological signal dominates the technical noise for RNA-Seq data.
    Tissue samples display higher levels of correlation
    for identical tissues in other studies than
    for other tissues in the same datasets.
    In other words, interstudy correlations for identical tissues
    are higher than intrastudy correlation between different tissues.
    The genes with the most variable expression across tissues
    or the most tissue-specific expression individually have
    the most meaningful influence on the correlation.\mybr\

    After a brief exposition of the proteomic data comparisons,
    I introduce a new proteomic quantification method,
    \PPKM\ (\Cref{ch:proteomics}).
    %on which I base the integration of the proteomics and transcriptomics.
    The \PPKM\ method allows me to quantify about twice as many proteins
    compared to usual methods.\mybr\

    Correct correlation levels
    between the expression of protein and mRNA
    in studies combining high-throughput transcriptomics and proteomics
    are sparse in the literature.
    However, I show that, for most tissues,
    we can observe good correlation levels,
    even when the samples have different biological and technical backgrounds
    as they have been independently sourced.
    Many genes share similar patterns of expression
    between the two biological layers,
    \eg\ genes that have a protein detected in a single tissue
    are more likely having an mRNA showing specificity for the same tissue.
    Additionally, three groups of genes present functional enrichments
    of biological processes.
    Genes having highly correlated protein and mRNA expressions across tissues
    are enriched in catabolic processes.
    Genes having highly anticorrelated expressions are enriched
    for ribosomes and ncRNAs regulation.
    Genes with a protein detected in a single tissue are enriched
    in signalling processes.\mybr\

    Overall, this thesis describes a global picture
    of the current consolidated knowledge
    we can extract from the joint study
    of public transcriptomic and proteomic data.
    Beyond confirming or improving observations reported in the literature,
    this work provides new insights
    into the ubiquitous and tissue-specific genes.
    To the best of my knowledge,
    this work has also established the most extensive list of genes
    with robust transcriptomic and proteomic expression across tissues and studies.
    Furthermore, it shows that joint study approaches can help the development
    of new methods, like the new proteomic \PPKM\ quantification method.
    Finally, the highlighting of distinct functional enrichment profiles
    for groups of genes across tissues and studies
    lays a framework for further research.\mybr\
    }
\end{singlespace}
