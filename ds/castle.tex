\section{Castle et al. dataset}
\label{ch:castleData}

%%\begin{description}
\begin{eqlist}
    \item[Type] Transcriptome
    \item[Library collection] One sample per tissue (pool of multiple donors per tissue)
    \item[Library preparation] \gls{PCR} amplification with specific oligonucleotides\\(whole RNA)
    \item[Technology] Single-end
    \item[Apparatus] Illumina GA-II sequencer
\end{eqlist}
%%\end{description}

This dataset has been published along with the \paper{\citetitle{castleData}}
    by \citet*{castleData} who were interested to explore
    with sequencing-based technology the whole RNA repertoire. At that time, the common
    employed amplification methods were inadequate to study non-polyadenylated
    \glspl{RNA} which is the case of most of the \glspl{ncRNA}.

    They designed the amplification step to avoid the \glspl{RNA}
    smaller than 50 \gls{nt} in order to deplete their pools of long
    \glspl{rRNA} \ribo{$28$S}, \ribo{$18$S}, \ribo{$16$S},
    \ribo{$12$S}. Unfortunately, most mature \glspl{miRNA} are not caught
    either while \glspl{rRNA} such as \ribo{$5.8$S} and \ribo{$5$S} are
    still amplified.

    They found coding genes could be highly tissue-enriched, e.g. \gene{PRM$2$}
    in \tissue{Testis}.\\They focused their study on the non coding part essentially
    on \glspl{snoRNA},~\glspl{scaRNA},\\~\glspl{scRNA},~\glspl{tRNA} and mitochondrial
    \glspl{RNA}.\\
    They finally conclude that \glspl{ncRNA} have higher and more tissue-specific
    expression patterns than \glspl{mRNA}.

    They used multiple-donors pooled tissues samples (purchased as total \gls{RNA}
    from Ambion (Austin, USA)).
    They prepared the libraries following a whole transcriptomic protocol \citep{Armour:2009}:
    specific oligonucleotides are used to amplify by \gls{PCR} nonribosomal \gls{RNA}
    transcripts.

    They generated an average of 50 millions sequence reads per tissue
    using an Illumina GA-II sequencer. They trimmed their original reads
    (50 \gls{nt}
    and 36 \gls{nt}) to 28 \gls{nt} that they released through EMBL archives
    (\ENA{ERP000257} and \ArrayExpress{E-MTAB-305}).

    Despite several limitations (lack of replicates, old technology, small reads),
    I used this dataset for two main reasons. It is the oldest available \Rnaseq\
    data I found that was performed on Human normal tissues and it is comprising
    \glspl{ncRNA}.

%\clearpage
