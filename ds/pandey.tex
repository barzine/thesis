\chapter{Kim et al. dataset}
\label{ch:pandeyData}



\cite{PandeyData} created the Human Proteome Map\footnote{%
\href{http://www.humanproteomemap.org/}{\small www.humanproteomemap.org/}} which
they released along their \emph{Nature} paper \paper{\citetitle{PandeyData}}.

For their study, they have processed 30 kind of histological normal human
tissues and cell line samples (17 adult tissues, 7 fetal tissues and 6
haematopoietic cell types). The samples were created from pooled samples of three
individuals (generally two males and one female).

Then, they prepared their libraries with a label-free method to quantify
as many proteins they could. They fractionated the samples to protein level by
\gls{SDS-PAGE} and then at peptide level by \gls{RPLC} to create 85 experimental
samples. Finally, they use state-of-art \gls{MS/MS} protocols
(with high-resolution and high accuracy \glspl{FTMS}:
Thermo Scientific Orbitrap instruments).
They generated about 25 million of (\gls{HCD})
high-resolution mass spectra which account for 2,212 \gls{LC-MS/MS} profiles.

While their effort to generate high quality raw data was highly appraise
by the scientific community, their processing
(identification and quantification) methods were
criticised (\eg\ see~\cite{Ezkurdia2014-qx}).

The raw spectra were retrieve from ProteomeXchange via the repository
\Pride{PXD000561}.
