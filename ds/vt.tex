\clearpage
\section{Brawand et al. dataset}
\label{ch:vtData}

%\begin{description}
\begin{eqlist}
    \item[Type] Transcriptome
    \item[Library collection] Biological replicates
    \item[Library preparation] PolyA-selected
    \item[Technology] Single-end sequencing
    \item[Apparatus]
\end{eqlist}
%\end{description}

In the corresponding article entitled \paper{\citetitle{VTpaper}},
 \citet{VTpaper}~collected 6 organs from 10 different vertebrates:
 9 mammalians (including Human) and a bird. They are focused on the
 evolution of the mammalian transcriptomes -- while there were existing studies
 on the matter, the sequencing approach was then creating new perspectives.
 They have biological replicates: one male
 and one female for the somatic tissue and two males for the testes samples.
 They used a
 polyA-selected protocol to prepare the libraries. Hence, the samples are largely
 enriched in protein coding genes.

 They generated an average of 3,2 billion reads of 76 base pairs per sample
 using an Illumina Genome Analyser IIx (single-end) and they released them
 through \gls{GEO} (accession number: GSE30352).
 I personally retrieved the data from
 \ArrayExpress{E-GEOD-30352}\footnote{ArrayExpress routinely imports
 datasets from \gls{GEO} on a weekly basis.}.


\clearpage
