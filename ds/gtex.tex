\section{The GTEx dataset}
\label{ch:gtexData}




The Genotype-Tissue Expression (\gls{GTEx}) project is funded by the NIH Common
Fund and aims to establish a resource database and associated tissue bank
for the study of the relationship between genetic variation and gene expression
and other molecular phenotypes in multiple reference tissues. The project was first
explained in a paper from the \cite{GTEx2013}. It consists to quickly collect
many tissues from postmortem donors so genotype-tissue expression analyses could
be done, notably \gls{eQTL} variants studies which study the modulation
of \gls{RNA} expression in function of \glspl{SNP}. The results of the
analyses are released through the GTEx portal (%
\href{http://gtexportal.org}{http://gtexportal.org}). The issue 6235 of
\emph{Science} comprises
many articles from this project. The most relevant to my work is
\paper{\citetitle{GTExTranscript}} from \cite{GTExTranscript}. While they study
the landscape of expression through the different tissues across the donors, they
put emphasis on the variation inter- and intra-individuals across the tissues.

As the project is quite ambitious and the collection and sequencing of the samples
are taking time, several freezes of the data have been released. My work is
including samples up to the fourth release of the pilot phase (v.1.4). This
release includes 54 tissue/cell type collected on 551 individuals.
The 3,276 libraries were prepared from whole \gls{RNA} extracts and then sequenced
with a paired-end protocol on Illumina HiSeq 2000/2500 sequencers which produced
an average of 80 million reads.

The raw data is available for privacy reasons only through controlled access via
\dbGaP{phs000424.v4.p1} (access number specific to the version of the data I used
in my study). While getting access can take time, in principle every request for
academic research should be granted.

\clearpage
