\section{Kuster et al. dataset}
\label{ch:kusterData}


In their human proteome approach,
\cite{KusterData} combined newly generated \gls{LC-MS/MS} spectrum
data (about 40\%) with already published one
(either from their colleagues or accessible through repositories ---
for the remaining 60\%).
The data comprised 16,857 experiments involving tissues, body fluids and cell
lines. They used all the data they could access from \gls{PTM} to affinity
purification studies.

They reprocessed the whole collection of spectra to maximise proteome coverage
and make it available through their own repositories: ProteomicsDB\footnote{%
\href{https://www.proteomicsdb.org/}{www.proteomicsdb.org/}}.

The subset of data considered in this study is also
known as the (protein) Human BodyMap
which is the part that was generated (1,087 \gls{LC-MS/MS} profiles)
by the Kuster lab itself.
It comprises 48 experiments covering 36 tissues
(adult and fetal) and cell lines.
They generated about 14 million of (\gls{HCD}/\gls{CID}) spectra from Thermo
Scientific instruments.

The raw data was downloaded from \Proteomicsdb{PRDB000042}.


\clearpage
