\section{Uhlen et al. dataset}
\label{ch:uhlenData}

%%\begin{description}
\begin{eqlist}
    \item[Type] Transcriptome
    \item[Study object] 32 tissues
    \item[Sample status] fresh frozen
    \item[Library collection] \emph{Biological} and \emph{Technical} replicates
    \item[Library preparation] PolyA-selected
    \item[Technology] Illumina Paired-end sequencing
    \item[Instrument] HiSeq 2500 and HiSeq 5000
\end{eqlist}
%%\end{description}


The first version of this dataset (\ArrayExpress{E-MTAB-1733}) was published
as a part of \paper{\citetitle{Uhlen2014}} by \citet{Uhlen2014}. Then an extended
version (\ArrayExpress{E-MTAB-2836}), with new samples and 5 new tissues,
was released along with \paper{\citetitle{Uhlen2015}} \citep{Uhlen2015}.
Both papers are part of the
\href{http://www.proteinatlas.org/}{Human Protein Atlas}\footnote{%
\href{http://www.proteinatlas.org/}{http://www.proteinatlas.org/}}.
Uhlén et al.\ have created
an atlas revolving mostly around the spatial distribution of the proteins through
the Human body. They use many approaches and techniques which also include \Rnaseq.

They found that almost half of the proteins are expressed in all analysed tissues
(with an enrichment for the metabolism enzymes).

There are \emph{technical} and \emph{biological} replicates for the 32 tissues.
Except very few cases, the somatic tissues have male and female donors.

The 200 libraries have been prepared following a polyA-selected protocols and
have been sequenced (paired-end) with an Illumina HiSeq 2000 or 2500.

I started to work with the first version, and when the extended version was
released, I included the new samples and tissues to my work.

\clearpage

