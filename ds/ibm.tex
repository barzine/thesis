\section{Illumina Body map dataset}
\label{ch:IBMData}

%\begin{description}
\begin{eqlist}
    \item[Type] Transcriptome
    \item[Library collection] ``Technical'' replicates (sequencing methodology)
    \item[Library preparation] PolyA-selected
    \item[Technology] Paired-end (and single-end) sequencing
    \item[Apparatus]
\end{eqlist}
%\end{description}

This dataset has been first created by Illumina mostly to advertise its most
recent technology improvement at that time: the paired-end sequencing.  Until
then, all the sequencing was done from only one end of the \gls{DNA} (or
\gls{cDNA}) fragments. With a tweak in the library preparation protocol, Illumina
introduced the paired-end sequencing which present many improvements over the
previous protocol (single-end). One particular enhancement, the detection of
rearrangements, allows the detection of gene fusions and novel splicing isoforms.


This dataset has been first created in 2010 and released in
2011\footnote{See: \citetitle{ibmEnsembl} - \cite{ibmEnsembl}} by Illumina
mostly to advertise its most recent technology improvement at that time:
the paired-end sequencing.
Until then, all the sequencing was done from only one end of the \gls{DNA} (or
\gls{cDNA}) fragments. With a simple change in the library preparation protocol,
Illumina introduced the paired-end sequencing which present many improvements
over the previous protocol (single-end). One particular enhancement, the
detection of rearrangements, allows the detection of gene fusions and
novel splicing isoforms. For this reason, most of the following transcriptome
studies based on \Rnaseq\ are using paired-end sequencing.

The first published paper to analyse this dataset was done by
\citet{ibmrelatedpaper}: \paper{\citetitle{ibmrelatedpaper}}.
It was referenced many times since then as it was for a couple of years
the most extensive freely available \Rnaseq\ dataset of human tissues.

It comprises 16 tissues (one donor per tissue), which were prepared with a
polyA-selected library preparation protocol and then have been sequenced once
with a singled-end protocol and then a second time with a paired-end one. There
are some added libraries which have been created by mixing together the 16 tissues.
While each sample has been sequenced twice and that we have in principle
\emph{technical} replicates, these are not ``regular'' technical
replicates.
The sequencing was performed with an Illumina HiSeq 2000 and the reads were
released through \ArrayExpress{E-MTAB-503} (\ENA{ERP000546}).

\clearpage
