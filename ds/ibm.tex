\chapter{Illumina Body map dataset}
\label{ch:IBMData}

%\begin{description}
\begin{eqlist}
    \item[Type] Transcriptome
    \item[Library collection] ``Technical'' replicates (sequencing methodology)
    \item[Library preparation] PolyA-selected
    \item[Technology] Paired-end (and single-end) sequencing
    \item[Apparatus]
\end{eqlist}
%\end{description}

This dataset has been first created by Illumina mostly to advertise its most
recent technology improvement at that time: the paired-end sequencing.  Until
then, all the sequencing was done from only one end of the \gls{DNA} (or
\gls{RNA}) fragments. With a tweak in the library preparation protocol, Illumina
introduced the paired-end sequencing which present many improvements over the
previous protocol (single-end). One particular enhancement, the detection of
rearrangements, allows the detection of gene fusions and novel splicing isoforms
