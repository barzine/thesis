\clearpage
\chapter{Integration of Transcriptomic data and Proteomics}

\fixme{Where it has been published or not, and who has done the analsys \ldots}
Collaborators: Nuno, James, Alvis and Jyoti\\
What-Who: correct discussion

\section{Introduction}

\begin{comment}
    here should be explain all the reasons and the challenging of why this is important
    Here some of the reasons: Transcriptomics and Proteomics are not the same range (technology bias)
    It is hard to say when it is NOT correlated if this comes from a technical problem or regulation.

    Workflows a lot better established in transcriptomics than Proteomics (annotation, mapping)

\end{comment}

The core of central dogma of molecular biology i.e.\ one \DNA\ (coding) gene will
be transcribed as one \mRNA\ transcript and will be in turn translated in one
protein still holds true even though we now know the truth is not as linear as it
could have been believed at first and many regulation processes are implied before
and after the transcription or the translation steps.
\fixme{can add a plot e.g. Konrad and develop a bit more the concept here}

However, as it has still to be proved that proteins can be produced directly
from \DNA, it is assumed that each protein is created by translation, hence studying
Transcriptomics and Proteomics together for the same conditions should give us
insights on these regulations processes.

Whereas it was implicitly assumed for a long time that for similar conditions
there should be a proportional relationship between Transcriptomics and Proteomics,
many studies \fixme{add references, see TAC report 3} done in cells have
failed to show high correlation between the two biological layers.
From then, while there are still studies done jointly on Transcriptomics and
Proteomics, the focus is greater either on the presence or absence of a
protein/transcript in a specific condition or if both transcript and protein
are differential expressed between two conditions in the same way.
There is not more
\fixme{'Big' \ldots there is an academese idiom for this I am sure}
effort to try linking directly their actual expression.
These studies however are mostly done at cell levels.
While cell populations could have external factors impacting their overall
expression, tissue samples should be driven by their function.
As such we expect that what makes a liver a liver and what makes a heart a heart
would overcome most of the technical variability.


Previous studies did not managed to show correlation between proteomics and
transcriptomics above 0.5 (Pearson correlation).\fixme{how to show properly pearson cor coef}

Here even though I have only access to independent data, the average correlation is above 0.5.


\section{Results}

While the pool of protein coding transcripts is about 60 to 70 \% similar from one tissue to another, the protein pool is quite different.

In the recent past few years (2014) two large proteomics assays focusing on normal
human tissues have been released for the first time.
\fixme{add citation for Pandey and Kuster}
Until now, there was not such availability of large-scale and extensive tissues
both on transcriptomic and proteomic layer to jointly study the genes
translation into proteins across a consequent set of tissues
at the same time. Alike the \dataset{Uhlén et al.} dataset and the \dataset{Gtex}
these datasets have not the same scope of tissues. However, the overlap of
studied tissues is enough to draw an overview of the current technology.

To compare the expression at proteomic and transcriptomic levels


\subsection{An extended protocol of identification and quantification of proteins}
\fixme{This has to be actually one of the last part of this chapter:
as I have the conservative method results to compare with,
I can prove that the extended method is probably as good since
the overall pictures doesn't change}


The available datasets and choosing Uhlen and Pandy for further analysis – the set of common tissues and common genes (i.e., gene/protein mapping)
Pandey quantification (done by collaborators – Jyoti and James)
Defining when a gene is expressed, ranking the expression
Tissue correlation
Tissue specific genes/proteins – how consistent they are in RNA and proteomics?
Gene correlation
Correlated and uncorrelated genes, what is specific to each group
functional groups of genes
Pseudo-gene expression




